Famed investor Charlie Munger once stated “Show me the incentive, I’ll show you the outcome” [60]. To understand commitment of firms to empowerment, incentives have to be analyzed. In this context that means cost and benefit to the firm to comply. This chapter outlines a theoretical framework to understand the causal relationship between B-BBEE policy and firm performance. What drives the relationship? Although the measurement of specific drivers of this relationship is outside the realm of this study, drivers are discussed to provide the reader with a richer understanding of the causal relationship between B-BBEE policy and firm performance. These drivers are instrumental in developing hypotheses to answer the research question. Hypotheses will be presented in the next chapter, Methodology.

This study identified three broad categorization of drivers of the relationship between B-BBEE and firm performance; signalling, compliance and productivity. Signalling entails the impact of B-BBEE on firm performance caused by the communication to the broader community of B-BBEE compliance. Compliance entails the direct benefit in terms of market access and direct costs in terms of cost incurred to comply with the B-BBEE targets. Finally, productivity entails the productivity effect B-BBEE compliance could have on firm performance.

These categorizations are not set in stone. Nor were they coined by the scientific community. These categorizations were created by the writer of this study to provide the reader with handles to better understand the drivers. These categories of drivers are not mutually exclusive, they mostly overlap. 
\section{B-BBEE and signalling effects}
The signalling effect of B-BBEE refers to the effect that broadcasting of B-BBEE compliance to the wider public has on firm performance. Signalling drives firm performance positively or negatively, depending on the commitment of a firm to the underlying objective of B-BBEE, namely to reduce wealth inequality within the South African society. Signalling, as defined in this study,  serves as an umbrella term for Corporate Social Responsibility (hereafter, CSR) and Fronting.
\subsection{Corporate Social Responsibility}
A firm committed to B-BBEE, according to van de Merwe and Ferreira, is exhibiting a form of CSR [7, p547]. Nowadays, CSR has moved from the ideology of ‘doing good is the right thing’ to reality [54, p1]. CSR is considered a necessary determinate for a firm portrayal in society by the way a firm incorporates ethical and social values in its business model [54, p1]. A firm that internalizes good practices can generate external effects such as positive reputation which can be monetized as a competitive position to put a firm in position to gain higher profits [34, p376]. Vasquez et al argue that firms advertise their good practices and as a result it induces consumers goodwill leading to firm support [34, p376]. In a similar context, Alessandri et al explains that firm reputation increases through positive media coverage because of a firm's engagement with B-BBEE [24, p9]. The paper by Mehta explains that firms that display good behavior on the basis of B-BBEE signals good management and therefore gains trust of investors [39, p58; 39, p65]. Firms can send signals to society that it is being socially responsible by selling a portion of its shares to Black people or placing a Black person as manager to empower Black disadvantaged groups [24, p9]. Local retailers purchasing and selling consumer goods in rural Black townships, acknowledge suppliers reputation through media or other advertising forms, and as a result opt to purchase supplies from the firm which are deemed as doing good in society [24, p9]. From a consumer’s perspective, the disenfranchised Black people are the group of people in South Africa with a higher propensity to spend. The majority of Black people, living on low costs in rural areas, have a relative higher portion of disposable income [24, p9]. Rural dwellers are the ones most affected by wealth inequality. As result, their consumer behavior could express their values regarding empowerment in their consumption pattern [34, p378]. This does not imply that the rest of society will consume in the same manner because some citizens may disregard a firm’s good intentions as part of their purchasing decisions [34, p379]. However, spendings made by those at the bottom of society can have a significant effect on a firm's performance. 
\subsection{Fronting}
A firm with no commitment can attempt to replicate the reputational dividend. In the context of B-BBEE this is referred to as “fronting” [24, p9-10; 27, p87; 39, p58;]. Fronting is defined as an act that undermines the objectives of B-BBEE [29, p26; 41, p17]. Fronting is classified under three categories namely; window dressing, benefit diversion and opportunistic intermediaries [41, p17; 43]. Window dressing occurs when a Black person is appointed managerial positions but are unable to participate in the decision making process [41, p17; 43]. This implies that the Black person is not empowered but rather placed as a ‘front cover’ to show a firm's involvement in B-BBEE. Benefit diversion describes firms receiving economic benefits under the B-BBEE status, but not distributing the benefits according to the B-BBEE Act [41, p17; 43]. Finally, opportunistic intermediaries are non-compliant firms in the supply chain that take advantage of compliant firms high credentials to get business agreements [ 41 p17; 43].  The practise of fronting came to the forefront in South Africa after the 1998 stock market crash that revealed high levels of debted among Black-owned firms [4, p18]. To set up BEE deals White owned firms offered loans to Black firms who lacked capital to purchase shares [6, p22]. It was thought that the payment of the loan could be serviced by the dividends made on the shares, however the firms were not generating profits [6, p22; 4, p18]. Tshetu states that such BEE deals were viewed as risky and market did not react positively to such deals [6, p22]. Furthermore these deals were not empowering in terms of ownership, as it was not sustainable, it was rather a partnership deal to get access to business licenses [6, p22-23; 4, p18]. A firm’s value would only be destroyed by allowing such transactions with partners that had no capital [6 p28]. Tshetu argues that, “this process demonstrates lack of honesty, a strategy not applied in good faith and unwillingness to participate in economic reform” [6, p28]. This was one of the reasons for the setup of a BEE Commission and fronting became a criminal offence [4, p18; 94]. The BEE Commission has the authority to investigate fronting practices. In fact, firms have been found guilty of fronting were made liable to pay fines or imprisonment [41, p20]. According to Davis (1973), South African firms ‘gradually sink into customer and public disfavour’ if they do not display social responsibility [7, p547]. Therefore, if B-BBEE has been mostly a result of fronting, then a negative relationship between B-BBEE and firm performance can be expected.
\subsection{Findings from previous research on signalling, B-BBEE and firm performance}
The previous two paragraphs outline that the signalling effect of B-BBEE on firm performance depends on the perception the wider community has of the firm’s compliance to B-BBEE. If the firm is considered truly committed then the effect should be positive and vice versa. Several studies measured the signalling effect. The signalling effect describes the impact on firm performance based on released information relating to B-BBEE compliance. This can be analysed through event studies. In these studies the effect of B-BBEE deals on share price performance was measured. If there was a abnormal return, then the signalling effect exists. Alessandri et al. found that announcements of B-BBEE transactions exhibit significant positive abnormal returns using 3 day and 5 day event windows by analyzing 20 B-BBEE transactions from 1996 to 1998 [24, p19; 24, p13]. Wolmarans and Sartorius analyzed 125 B-BBEE transactions between 2002 and 2006 and found, using a set of different event windows, significant positive average abnormal returns [50, p186; 50, p187]. Mehta and Ward deviated from the typical event studies and evaluated the impact of a change in B-BBEE score on the publication date on share price return, dividing the analysis of events into upgrades of B-BBEE score and downgrade [27, p91]. Mehta and Ward studied 70 upgrades events occurring between 2009 and 2015 and observed significant positive abnormal returns [27, p91]. This study also observed significant negative abnormal returns over 24 downgrades observed over the same time period [27, p92].

Even though some of the studies, such as Alessandri et al. and Mehta and Ward in their analysis of downgrade used relatively limited sample size, the body of research over different time periods suggest that the short term effect of B-BBEE on share price return is significantly positive. This conclusion opposes the finding of Tsethu, mentioned in the previous paragraph. This contradiction could be explained by time period bias, or the fact that the event studies merely analyze a short time period whereas Testhu discusses the longer term effect fronting had on firm performance.

Unfortunately, none of the studies disclose industry bias in their research. A B-BBEE compliant firm is expected to benefit when it is dependent on government related industries, thus an upgrade in B-BBEE through positive returns. Whereas, B-BBEE compliant firms that are not dependent on government related industries is not expected to benefit.  It is expected that the market would respond more positively to an upgrade in B-BBEE score than an upgrade in B-BBEE score of a firm which operates in an industry relatively independent from the government. Recall from the section “Mechanics 2013 to 2019” that government entities are required to base their supplier selection upon B-BBEE score of the supplier. This means that an upgrade in B-BBEE score could increase the probability of revenue increase for such a supplier, to which the market should respond positively. For a firm operating in an industry relatively independent of government, an upgrade in B-BBEE score could receive a more muted response of the market as the upgrade is not as directly linked to revenue gains for the firm compared to a firm that is operating in a government related industry. In the industry bias, the overlap between the signalling effect and compliance effect is noticeable.

Mehta and Ward claim that the short term effect of B-BBEE can be linked to adjustment by investors to a ad hoc release of B-BBEE  information on the long term  economic viability of a firm [27, p95]. This relates to a firm’s genuine commitment to empowerment of Black people. The market responds to CSR. Despite this claim, the short term effect whilst significant, could also be linked to short term noise or fronting. Consider a case of fronting. A firm receives a upgrade in B-BBEE score, because the element ownership increased based on a B-BBEE transaction which transferred shares to Black people. As result, in this example, the market initially responds positively, generating significant positive abnormal returns. However, after the upgrade new information is released which proves that the B-BBEE transaction upon which the upgrade B-BBEE score was based was indeed a case of fronting, because the firm performed terrible on the other elements. The market would have to process this additional information. Logically by erasing the significant positive abnormal returns generated after the upgrade in B-BBEE score. The  initial response, which Mehta and Ward measure, does not capture the long term effect of the B-BBEE transaction. Mehta and Ward attempt to incorporate this second response by monitoring returns 180 days after the event [27, p89]. However, given the inert nature of B-BBEE legislation and the fact that B-BBEE scores get adjusted annually, it is difficult to accept that 180 days was a sufficient window to incorporate a realization of the market of incorrect scores. In other words, measuring a firm being truly committed, i.e. CSR, could require measuring the response of the market to their broadcasting their commitment on a ongoing basis. This is not captured in any of the event studies.
\subsection{Summary and conclusion}
The signalling effect covers two concepts, corporate social responsibility (CSR) and fronting. In South Africa CSR can be seen in the form of B-BBEE. It can be expected that CSR has a positive signalling effect on society. Firms are regarded as doing right and empowering Black people therefore drives support for a firm, thereby increasing firms profits. On the other hand a firm’s profit may decrease, as fronting practices are not favored by society. Fronting does not empower Black people because they are simply used as front covers but are not involved in a firm’s activities. Fronting can be thought of as half compliance and CSR as full compliance. Previous studies attempt to measure the effect of BEE deals on share price return. This essentially measures the signalling effect. Majority of the studies find a positive effect of signalling on share price return. However, these studies do not correct for industry bias, nor do they measure whether the signalling effect was caused by CSR or fronting.      
\section{B-BBEE and compliance effects}
The compliance effect of B-BBEE drives firm performance positively or negatively, depending on the extent to which revenue increase attributable to B-BBEE compliance outweighs the costs of B-BBEE compliance. The generic scorecard that was created specifically for firms with >50 million rand turnover per annum determines if large firms are compliant with B-BBEE or not. Compliance, under the B-BBEE policy is defined by a large firms BEE level. Firms are considered compliant if their BEE level are between 1 and 8, 8 being 30 percent and above threshold and below the 30 percent the firm is not compliant [46, p9].
\subsection{Compliance benefits of B-BBEE}
Government suggests the higher the B-BBEE score of a firm the higher its firm performance [4, p46; 4, p10; 4, p18]. It is important to note that private firms and public entities were not obliged to report B-BBEE compliance under the 2003 Act, however under the amended Act of 2013 it was required that all public entities and JSE listed companies report their compliance annually [4, p36]. Compliance towards B-BBEE was promoted through direct and indirect incentives for private firms. Directly, if firms wish to conduct business at hierarchical levels with government and other public entities then it must comply to B-BBEE [7, p548; 46, p9]. The advantage of B-BBEE compliance is that firms can obtain a quotas, business license that gives a firm access to tender for government contracts and gain access to market whereby they can enter into contracts with other public entities [7, p546; 46, p9; 23, p9]. B-BBEE compliant firms that are awarded government tenders carries large value, meaning government payout would range more or less between 1 to 10 million rand [58]. Indirectly, a supply chain may also be affected by compliance, under the preferential procurement (a B-BBEE element) regulation any firm supplying goods or services to the government or other public entities must ensure that the supplies purchased are from B-BBEE compliant firms [7, p546; 3,5 p9]. This is also known as the trickle down effect, were the rest of the firm's down a supply chain are persuaded and pressured to comply if they intended to do business together [47; 7, p546]. Disagreement on the above mentioned is found by Jack and Harris (2007) who argue that large firms (such as those in tourism sector) do not find it necessary to comply with B-BBEE because they do not rely on government contracts because bulk of the sales are made directly to the general public [4, p35]. Furthermore, the general public or small to medium enterprises makes purchasing decisions for goods or services based on price and quality and not on BEE compliance [4, p35].
\subsection{Compliance costs of B-BBEE}
Heavy costs are associated with complying to B-BBEE policy. some scholars suggest that costs of obtaining a high B-BBEE score outweighs the benefits [4, p2; 7, p548; 45, p232]. This implies efforts to comply with B-BBEE have negative side effects despite the pronounced benefits (implying to both long and short term effects). Although firm’s comply to B-BBEE and do tender for government job, they are required to pay a fee (depending on the job) to submit the tender document [59 p36]. This does not guarantee a firm will win the bid however, they will still incur a cost. The cost of compliance relate for example to BEE transactions which were executed against discount of market price, as discussed in the section “The first phase of BEE, BEE as the transfer of shares”. Furthermore, other elements of the B-BBEE scorecard are also associated with cost for a firm. Recall from “Mechanics 2013 onwards” section that the 2013 Codes of good practise dictate a firm spends at least 6\% of total payroll spending on developing skills for Black employees, 25\% of cost of sales ex. labor costs and depreciation must be spent in South Africa and at most 1\% of profit spent on socio-economic programs. These targets increase to cost of doing business for firms and if they are not compensate by either cost savings or efficiency gains, these result in a negative relationship between B-BBEE and firm performance.
\subsection{Findings from previous research on compliance, B-BBEE and firm performance}
As noted in the previous research, scholars have argued that the costs of B-BBEE compliance outweigh the benefits. For example, Mokgobinyane concludes that on several measures of firm profitability, the costs of B-BBEE compliance does not translate in higher profitability in 2007, 2010 and 2013. Mokgobinyane compared firms that comply with B-BBEE to firms that did not.

Mokgobinyane uses the B-BBEE aggregate score as the treatment variable, with control variables size, leverage, liquidity and industry [4, p46]. These variables are then tested with different dependent variables, Revenue, Net profit margin and return on equity [4, p48; 4, p49]. These three regression models are then tested separately for three years, namely 2007, 2010 and 2013 [4, p53]. Data availability limits the ability to generalize the findings of Mokgobinyane. Using only three specific years, the study makes it self vulnerable to selection bias. This is underlined by the fact that the new Codes of good practise released in 2013 was not included in Mokgobinyane’s research. Furthermore, the industry control variables that Mokgobinyane used were different for the different years. For example for 2010 and 2013 industry classifications Basic materials, Consumer Services, Financial and Industrial were used. For 2007, however, Technical, Industrial, Financial, Basic materials, Consumer services, consumer goods and health care industry classification were used. This could indicate sample size bias, i.e. the 2007 dataset was more dispersed compared to the 2010 and 2013 dataset. Acemoglu argues that the efficacy of B-BBEE aggregate score depends on the cost and benefits of a firm on being B-BBEE compliant [23, p26]. Therefore, control variables include fraction of shares held by the South African Public Investment Corporation and industry with industry charters [23, p27]. Acemoglu, similar to Mokgobinyane, relies on Empowerdex for B-BBEE aggregate score [23, p29]. The Acemoglu paper tests the impact of B-BBEE score on net profit margin for the period 2004-2006, yielding 159 observations [23, p29]. Acemoglu finds positive, but not significant, relationship between B-BBEE aggregate score and net profit margin [23, p32]. The sample size was rather small, focussing only on the 2004-2006 time period. This makes it difficult to generalize this study’s finding on the relationship between B-BBEE aggregate score and net profit margin. Also, although intuitively it makes sense to focus on drivers of B-BBEE cost and benefits to select control variables to prevent omitted variable bias, even this approach is not devoid of omitted variable bias. Considering the impact that the foreign funding stop had on the Apartheid regime (as discussed in the section “Apartheid era”), perhaps a wider definition should have been adopted, rather than only controlling for shares held by one institutional investor, in this paper the South African Public Investment Commission.

Merwe and Ferreira test the effect of B-BBEE aggregate score on share price return. Contrary to the challenges that Acemoglu and Mokgobinyane faced with selecting control variables, Merwe and Ferreira selected thoroughly tested control variables that impact share price return, complimented by a industry dummy vector. [7, p550]. Merwe and Ferreira find a significant negative relationship between B-BBEE aggregate score and share price return [7, p552]. However, Merwe and Ferreira note that their study could be subject to a bias due to the time period as Codes of good practise were revised in 2013 [7, p555]. Mehta and Ward compose 4 portfolios, based on their rank [27, p90]. Put straightforward, the top portfolio consists of the top 25\% B-BBEE scoring firms. Each quarter the portfolios were rebalanced to account for changes in B-BBEE score as well as changes in the sample size [27, p90]. Using this methodology, Mehta and Ward find a relationship between B-BBEE aggregate score and share price return [27, p95]. The portfolio with the top 25\% B-BBEE scoring firms, performed the worst. However, what is noteworthy is that Mehta and Ward did not create the portfolios industry neutral. This means that the portfolio with the top 25\% B-BBEE scoring firms could consist of firms all operating in a industry facing share price return decline. By not controlling for industry, Mehta and Ward were vulnerable to capturing industry effects as well as the effect of B-BBEE on share price return.

Although, the studies discussed above assign the relationship between B-BBEE on firm performance to compliance, the overlap between the drivers as defined in this study makes it difficult to assess the difference between compliance and productivity from the results found by Acemoglu and Mokgobinyane. For example, a B-BBEE compliant firm could have higher profitability. This could be due to revenue growth from gaining access to government contracts or productivity growth which increases cost efficiency.
\subsection{Summary and conclusion}
This section discussed compliance effect which can drive firm performance positively or negatively. It is important to note that non-compliance is disregarded in this thesis because non-compliance would be out of the scope of this research and does not fit with the research question. B-BBEE compliant firm’s are put in advantageous positions because they can conduct business with government that can increase a firm’s profit. However, tendering for government jobs do not provide certainty for winning the tender and a financial cost is incurred at the end for submitting tender documents. Moreover, financial cost of compliance with B-BBEE are attached to the elements in the scorecard. Firms are expected to spend on extras like development of skills and training. The next section discusses productivity effects which sheds light on pro and cons of related skilled and productive firms and people. Overlap between productivity and compliance make it difficult to measure the extent to which compliance drives the relationship between B-BBEE and firm performance. Nonetheless, previous studies on the compliance effect, B-BBEE and firm performance are not in consensus on the relationship between these variables, but the majority of the studies indicate a negative relationship. Their findings however are hampered by the  small sample sizes used. 
\section{B-BBEE and productivity effects}
The productivity effect of B-BBEE drives firm performance positively or negatively, depending on the extent to which economic renting existed in the Apartheid era and whether the extent to which market friction creates a barrier to dissolve economic renting. Productivity can be defined as a measure of efficiency for a individual, firm, equipment or system that are known as inputs which are used to generate outputs which are goods and services [57 p1]. Resources that are allocated efficiently should enhance productivity of a firm, thereby allowing productive firms to encounter uptake performance.
\subsection{B-BBEE and productivity gains}
Acemoglu et al. suggest that the apartheid regime created the most unequal society in world, because the South African society was economically and politically structured by White people that aimed to suppress Black people and extract resources [23, p11]. The White supremacists behavior was described as economic rent seekers, their behavior generated enormous misallocation of resources [23, p11].  Economic rent seeking essentially entailed that privileged White people positioned themselves in managerial position wherein their added value to the economic process was microscopic at best. Rather, privileged White people reaped the benefits from their black subordinates hard work that did add significant value to the economic process but Black subordinates did not reap much of the benefits of their work. Renting seeking is described to be unproductive because it destroy value by depleting valuable resources [51, p74]. The South African government intervention through BEE required White owned firms to reallocate resources of capital, land and labor [24, p8]. For example, Black people were skilled in agriculture however, they had no rights to own land and control capital [23 p12]. This meant companies needed to change from a situation where Black people were abused to a situation where Black people were beneficiaries (24, p8). The BEE Commission set objectives to increase access to production assets to ensure ownership participation and employment for Black people would increase (4 p21). Firms that are unproductive function under incorrect racial assumptions of Black people being less productive (23 p19). However, if Black people who are more qualified are hired then productivity increases resulting in more profit (23 p20). Due to market imperfections, government's interventionist policy of B-BBEE forces unproductive firms to hire skilled Black people and to sell their shares to Black people who are more productive then these firms could be run more efficiently (23 p20). Furthermore, B-BBEE compliant firms spend on skill and development, which means that Black people are given opportunities and encouraged to learn and be productive [56, p6]. It is expected that BEE-owned managers should portray entrepreneurial traits, skills and management qualities which should enable them to continuously innovate and create opportunities that will generate profits [6 p32]. In this scenario, B-BBEE would increase firm performance because labor production improves but also B-BBEE allows for the switch from non-productive to productive.
\subsection{B-BBEE and productivity losses}
Despite the positive side B-BBEE production effect there are also negative productivity effect, the productivity argument is widely criticized. BEE transfer of shares for ownership had only benefited few well connected politicians and as a result created a unproductive wealthy group of Black people that disregarded the poor [6, p76; 30, p301]. The transfer of shares had made them instantly wealthy thus reducing their need to be productive. Alternatively, scholars argue that Black people are unable to exploit unproductivity as poor rural dweller face friction. Three types of friction exist: First, Capital markets are inefficient - even if a Black person is more productive than a White person, the Black person will not have wealth to buy an asset and therefore needs to go to capital market. Second, market is not able to efficiently provide the public with good education. Third, racial stereotypes persist meaning Black people are still seen as inferior to White people so even if Black people have same education/qualification they will still not get the job [23, p12]. Black managers are perceived to be less educated and unable to manage a firm because of their educational background, thereby creating a risk factor for investors [24, p10]. Investors are usually White or colored South Africans, there are not many Black South African investors [24 p10]. Due the racial stereotypes that are made against Black people, investors may avoid B-BBEE compliant firms because these firms are less likely to be productive and are expected to run losses in the expected future [24 p10].
\subsection{Findings from previous research on productivity, B-BBEE and firm performance}
This study found no research that assigned the relationship between B-BBEE and firm performance solely to productivity gains. Nonetheless, as mentioned in the section “Findings from previous research on productivity, B-BBEE and firm performance” there is significant overlap between this study’s definitions of productivity and compliance.
\subsection{Summary and conclusion}
Productivity can drive a firm profit positively or negatively. B-BBEE allowed for the previously skilled but disadvantaged Black people to get access to capital and land. By re-allocating and having access to productive assets production can be regarded as more efficient because resources are used efficiently, thus, resulting in an increase in firm profitability. B-BBEE aimed to remove economic rent seekers that were not productive and used resources inefficiently. However, with the BEE Act, the transfer of shares benefitted a few well connected Black politicians. Their previous disadvantaged positions had changed and they had gained wealth which they had never had before, this could of incentivized them to become less productive. Albeit, B-BBEE was meant to empower Black people and it did by Black people taking up management positions in a firm, investors are less likely to invest in firms that have Black people as managers. This is due to the racial stereotypes that are made towards Black people being unable to manage a firm efficiently because Black people considered to have lower educational background. Such perceptions welcome non-investment motives and indirectly affects firm's profitability to decline.
\section{Summary and conclusion}
This chapter aimed to to understand the causal relationship between B-BBEE policy and firm performance. Three self-defined categories of drivers were identified; signalling, compliance and productivity. Signalling, compliance and productivity have both positive and negative effects that as such these three categories are considered to be the drivers of firm performance. The significant overlap between these three categories make it difficult to assess the potency of one particular driver on the relationship between B-BBEE and firm performance. The drivers can be subdivided in positive drivers and negative drivers. Despite the entanglements between the drivers and both positive and negative effects of the drivers, scholars do find that these drivers impact the relationship. Therefore the categorization, again, should be regarded as a handle to broadly understand the drivers between B-BBEE and firm performance. The specific driver of the relationship between B-BBEE and firm performance falls outside the scope of this study. The findings on the relationship between B-BBEE and firm performance are not in consensus.

Signalling should have a positive effect when firms are committed to the empowerment of Black people. These are CSR firms. On the other hand, fronting is used to game and increase firm performance. However, in the longer term fronting has a negative effect on firm performance, especially considering that fronting has become a criminal offense. Studies on signalling indicate that signalling of B-BBEE compliance has a positive effect on firm performance. However, these studies do not correct for industry bias, nor are they able to distinguish between CSR and fronting as the studies are mostly short term event studies. Compliance with B-BBEE policy has benefits and costs. The advantage of B-BBEE compliance is that firms can obtain a access to market. However, the costs to comply to B-BBEE, increased spending on staff, selling shares at a discount to market prices, detract from firm performance. The overlap between compliance and productivity is that however, the cost of training could increase productivity of the firm which in turn could increase firm performance. Vice versa, B-BBEE compliance could also fuel productivity losses in the case where B-BBEE share transfer only benefited few well connected politicians and as a result created a unproductive wealthy group of Black people. The overlap between productivity and compliance makes it difficult to assess the impact these drivers had individually. Rather, scholars, rightfully, analyze the relationship B-BBEE and firm performance. The bulk of the studies indicate a negative relationship between B-BBEE and firm performance. These findings are difficult to assess the long term effect between B-BBEE and firm performance as time horizons of only a few years are used.