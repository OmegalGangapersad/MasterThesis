Famed investor Charlie Munger once stated “Show me the incentive, I’ll show you the outcome” \cite[]{N59}. To understand defensive strategy of firms to empowerment, incentives have to be analyzed. In this context that means cost and benefit to the firm to comply. This chapter outlines a theoretical framework to understand the causal relationship between B-BBEE policy and firm performance. Firm performance is discussed, revealing that profitability is the appropriate proxy for firm performance. Thereafter costs and benefits are discussed to provide the reader with a richer understanding of how B-BBEE policy affects profitability, explaining why causal relationship between B-BBEE policy and firm performance should exist. Next, models to measure the relationship are discussed, sourced from previous research on the relationship. How the relationship is measured, could impact the nature and the strength of the relationship. Together, the historical background discussed in the previous chapter, complemented by understanding of firm performance as profitability, the conceptual analysis of cost and benefits of B-BBEE policy for firms, and findings from previous research on the relationship form the theoretical framework upon which, at the end of this chapter, hypotheses will be formed.

\section{Defining firm performance}
To understand the relationship between B-BBEE policy and firm performance, one needs to adopt a proxy for firm performance. The concept of firm performance is broad, it could  refer to, amongst others, profitability, growth, productivity, efficiency and competitiveness \cite[p96]{N41}. Traditionally, firm performance was defined by firm type. For-profit firms focused, quite evidently, on profit measures as the measure of firm performance. Contrary, non-profit firms targeted a broader spectrum of objectives, such as customer satisfaction, as measurements for firm performance \cite[p96, p102]{N41}. Non-profit entities were for example government sponsored entities which were mainly focused on service delivery. For example, a nationalized power utility was mostly focused on providing power to a nation at minimum costs. However, in general these non-profit firms were loss making and thus inefficient. This was particularly well reflected by the capitulation of the Soviet Union, which  led to mass privatization in the late 1980s. In South Africa, the end of Apartheid heralded a wave of privatization. These waves of privatizations proved that business continuity measures, such as profitability, regardless of firm type should not be ignored as firm performance. Traditional for profit firms, on the other hand, were also under pressure. The pressure for these firms arose from social pressure as response to the  myopic focus on profitability by for profit firms. Recall from the Contextualization chapter that Chase Manhattan, under civil pressure from within the United States, stopped providing funding to South Africa \cite[p324]{N32}. Firms were reminded that their firm performance expanded beyond profit objectives. However, this also proves that a firm ignoring their broader responsibility will eventually find its profitability diminishing. Therefore, ultimately profitability is the appropriate measure for firm performance.

Profitability itself can be operationalized through various measurements. For example, net income, is the bottom line profit that firms report on predetermined dates over different time horizons (annually, quarterly or semiannually). Clark et al. (\citeyear{N16}, p11) argue that social responsibility, should result in higher net income, whether it be through increased efficiency or increased demand. Put straightforward, a firm that operates socially responsible with regards to all stakeholders will be more popular with consumers, which will increase revenue and ultimately result in higher net profit for the firm. As such, demand for the firm’s business model should increase, creating a universe of firms that caters to all stakeholders. Alternatively, a firm that operates socially responsible would, for example, use less  resources to create finished products and therefore have less raw material costs and higher profit. Mokgobinyane (\citeyear{N4}, p49) used firm profitability measures such as net profit margin, return and equity to measure the relationship between B-BBEE compliance and firm performance . Mokgobinyane (\citeyear{N4}, p7) states that the advantage of using these proxies is that they are, unlike the proxy share price performance, isolated from general market movements.  Acemoglu et al. (\citeyear{N23}, p29) tests the impact of B-BBEE aggregate score on profitability as well, using net profit margin as the dependent variable . However, measures such as profitability can be manipulated through non-cash movements. For example, a firm could change depreciation scheme by increasing asset life, which would not reflect any improvement of the firm’s underlying business but the firm would incur lower depreciation and thus higher profitability. Both the studies of Acemoglu and Mokgobinyane do not control for such accounting gimmicks.

Alternatively share price performance is a reflection profitability which should correct for accounting gimmicks. Companies with higher profitability have been observed to attract more investment flows which drive share price performance \cite[p59]{N4}. Investment flows can be viewed as, a reflection not of current profitability of the firm, but projected future profitability of the firm. Conventional investment decisions typically are based on discounted future cash flows. The forward looking character of investment flows should allow it to smoothen capital expenditures, such as B-BBEE related costs. Capital expenditures however could have ad hoc impact on backward looking profitability measures such as net profit margin and return on equity. Investment flows also impact the cost of funding of a firm. For example, according to research, credit ratings are lower for firms that have superior sustainability scores lowering the cost of debt \cite[p24]{N16}. Credit ratings are obtained by independent credit agencies that examine the creditworthiness of a firm, the better the credit rating the lower interest rate a firm has to pay. Argued from an alternative vantage point, institutional investors such as pension funds are increasingly obliged by regulators and governments to incorporate ESG score into their investment decision-making process, will drive up share price. Within the realm of this study, the South African sovereign wealth fund Public Investment Corporation, must invest in B-BBEE compliant firms \cite[p27]{N23}. This increases the share price of B-BBEE compliant firms strengthens the firm. A firm could for example leverage the share price inflation to raise money from the public equity markets which would allow the firm to achieve economies scale of advantages. The majority of previous research of the impact of B-BBEE on firm performance used share price return as the proxy variable for firm performance (\citeauthor{N24}, \citeyear{N24}, p16; \citeauthor{N23}, \citeyear{N23}, p29 ; \citeauthor{N27}, \citeyear{N27}, p549 ; \citeauthor{N7}, \citeyear{N7}, p551). 
\subsection{Summary and conclusion}
This section reviewed several proxies for firm performance. It found that the responsibility of firms exceed profitability, but that in essence profitability is the core principle for firms. The larger responsibility of firms is encapsulated in profitability measures, as firms that do not consider their broader responsibility could be punished by diminished profitability. Thus, profitability is a good proxy for firm performance.

Profitability as it relates to previous research on the relationship between B-BBEE policy and firm performance, is operationalized through accounting measures of profitability or share price return. The advantage of accounting measures of profitability is that it isolates profitability from external noise. However, accounting measures such as net profit margin and return on equity can be manipulated, thereby becoming an ill reflection of profitability. Share price return do not suffer from these manipulations. Further share price returns capture future profitability discounted to current market price. Share price returns also capture investment flows of institutional investors that appreciate a firm’s compliance to B-BBEE. Therefore, this study, as well as most previous research, conclude that share price returns are the most appropriate operationalisation of firm profitability.
\section{Costs and benefits of B-BBEE policy for firms}
Profitability, whether measured through accounting measures or expectations of future profitability discounted to current price, are a function of revenue and costs. The reluctance of firms to wholeheartedly embrace the goal of Black Economic Empowerment generally and B-BBEE policy specifically, suggests that costs to comply B-BBEE policy could be substantial. This section explores the benefits and costs to establish that that there is a relationship between B-BBEE policy and firm performance. 
\subsection{Benefits of B-BBEE policy for firms}
A firm committed to B-BBEE, according to van de Merwe and Ferreira (\citeyear{N7}, p547), is exhibiting a form of CSR . Nowadays, CSR has moved from the ideology of ‘doing good is the right thing’ to reality \cite[p1]{N54}. CSR is considered a necessary determinate for a firm portrayal in society by the way a firm incorporates ethical and social values in its business model \cite[p1]{N54}. A firm that internalizes good practices can generate external effects such as positive reputation which can be monetized as a competitive position to put a firm in position to gain higher profits \cite[p376]{N34}. Vasquez et al. (\citeyear{N34}, p376) argue that firms advertise their good practices and as a result it induces consumers goodwill leading to firm support. In a similar context, Alessandri et al. (\citeyear{N24}, p9) explains that firm reputation increases through positive media coverage because of a firm's engagement with B-BBEE. The paper by Mehta and Ward (\citeyear{N39}, p58, p65)explains that firms that display good behavior on the basis of B-BBEE signals good management and therefore gains trust of investors. Firms can send signals to society that it is being socially responsible by selling a portion of its shares to Black people or placing a Black person as manager to empower Black disadvantaged groups \cite[p9]{N24}. Local retailers purchasing and selling consumer goods in rural Black townships, acknowledge suppliers reputation through media or other advertising forms, and as a result opt to purchase supplies from the firm which are deemed as doing good in society \cite[p9]{N24}. From a consumer’s perspective, the disenfranchised Black people are the group of people in South Africa with a higher propensity to spend. The majority of Black people, living on low costs in rural areas, have a relative higher portion of disposable income \cite[p9]{N24}. Rural dwellers are the ones most affected by wealth inequality. As result, their consumer behavior could express their values regarding empowerment in their consumption pattern \cite[p378]{N34}. In this cases B-BBEE compliance would positively impact firm performance.

B-BBEE compliance also benefits firms by providing access to business with government entities. B-BBEE compliant firms that are awarded government tenders carries large value, meaning government payout would range more or less between 1 to 10 million rand \cite[]{N58}. Indirectly, a supply chain may also be affected by compliance, under the preferential procurement (a B-BBEE element) regulation any firm supplying goods or services to the government or other public entities must ensure that the supplies purchased are from B-BBEE compliant firms (\citeauthor{N7}, \citeyear{N7}, p546; \citeauthor{N3}, \citeyear{N3}; \citeauthor{N5}, \citeyear{N5}, p9). This is also known as the trickle down effect, were the rest of the firm's down a supply chain are persuaded and pressured to comply if they intended to do business together (\citeauthor{N47}, \citeyear{N47}; \citeauthor{N7}, \citeyear{N7}, p546). This indicates that even firms that do not directly deal with government entities could increase firm performance by complying to the B-BBEE policy. Nonetheless, the market access benefit of B-BBEE policy is constrained by sector. For example, Jack and Harris (\citeyear{N4}, p35) note that large firms operating in non government facing sectors (such as those in tourism sector) do not find it necessary to comply with B-BBEE because they do not rely on government contracts as of their revenue is generated by servicing the general public. This suggests that B-BBEE compliance would positively impact firm performance of firms operating in particular (government-related) sectors. On the other hand, compliance B-BBEE could also be viewed as a prerequisite of continuing business. In this sense, for these sector for which B-BBEE compliance provides access to government related business, could be viewed as increasing costs to continue to operate. Consider a firm that has a long standing relationship with government entities. As government increase targets, this firm has to incur the cost relating to B-BBEE compliance, just to maintain the relationship. Thus, the perceived benefit of market access through B-BBEE compliance could also just pose a cost. This notion is confirmed by Lindsay, who argues that firms viewed B-BBEE policy as a form of taxation \cite[p187]{N30}.

Finally, B-BBEE compliance could also benefit firm performance through productivity gains. This relates to the elimination of rent seeking behaviour within the management of firms through management control targets set in the B-BBEE policy. Economic rent seeking essentially entailed that privileged White people positioned themselves in managerial position wherein their added value to the economic process was microscopic at best. Rather, privileged White people reaped the benefits from their black subordinates hard work that did add significant value to the economic process but Black subordinates did not reap much of the benefits of their work. Renting seeking is described to be unproductive because it destroy value by depleting valuable resources \cite[p74]{N51}.  If Black people who are more qualified are hired then productivity increases resulting better firm performance \cite[p20]{N23}. In this case B-BBEE policy would increase firm performance.
\subsection{Costs of B-BBEE policy for firms}
Contrary, there are also aspects of B-BBEE compliance for firms which could decrease firm performance. For example, compliance by share transfer did in particular cases decrease firm performance. Tshetu (\citeyear{N6}, p22) states that share transfers from firms to Black people sold at premium to market prices were viewed as risky and as result share price of these firms would collapse. A firm’s value would only be destroyed by allowing such transactions with partners that had no capital \cite[p28]{N6}. This was one of the reasons why fronting became a criminal offence \cite[p18]{N4}. The BEE Commission has the authority to investigate fronting practices. In fact, firms have been found guilty of fronting were made liable to pay fines or imprisonment \cite[p20]{N41}. If firms complying to B-BBEE policy by selling shares is perceived as fronting or disingenuous then compliance of firms to B-BBEE policy could decrease firm performance.

Even if firms were not perceived disingenuous in their pursuit of B-BBEE compliance, costs of B-BBEE policy compliance remain. For example to BEE transactions which were executed against discount of market price, dilute existing shareholders value  and therefore present as a detraction of firm performance. Furthermore, recall from the section B-BBEE policy era that the targets for B-BBEE compliance set in the 2013 Code of Good practise should pose significant costs to firms. At least 6\% of total payroll should be spend on developing skills for Black employees, 25\% of cost of sales ex. labor costs and depreciation must be spent in South Africa and at most 1\% of profit spent on socio-economic programs. These targets increase to cost of doing business for firms therefore compliance to B-BBEE policy could decrease firm performance.

Despite the positive side B-BBEE production effect there are also negative productivity effect, the productivity argument is widely criticized. BEE transfer of shares for ownership had only benefited few well connected politicians and as a result created a unproductive wealthy group of Black people that disregarded the poor (\citeauthor{N6}, \citeyear{N6}, p76; \citeauthor{N30}, \citeyear{N30}, p301). Rather than eliminating rent seeking behaviour, this indicated that rent seeking behaviour was sustained. Black managers are perceived to be less educated and unable to manage a firm because of their educational background, thereby creating a risk factor for investors \cite[p10]{N24}. In these cases compliance of firms to B-BBEE policy could decrease firm performance.
\subsection{Conclusion and limitation}
This section provided a conceptual overview of costs and benefits related to B-BBEE compliance for firms, rather than measuring the exact cost and benefits to determine which cost or benefit dominate. Nonetheless, share price movement as result of B-BBEE share transfer indicate that at minimum B-BBEE policy does affect firm performance (with share price as a proxy for firm performance). Alternatively, direct benefits to firms to comply include access to government contracts. This also proves that firm performance is affected by B-BBEE policy compliance. Further, this section indicates that compliance to B-BBEE policy also affects efficiency through productivity effects, thereby solidifying the notion that B-BBEE policy should impact firm performance.

The nature of this relationship is not clear at this point of the study as compliance also entails significant costs which decrease profitability. However, exploring the direct benefits and costs of B-BBEE policy does hint that the relationship between B-BBEE policy and firm performance differs across sectors.

This section is limited as it does not identify which cost or benefit dominates, this is outside the scope of this study. Further, it is difficult to distil the driving cause of the relationship between B-BBEE policy and firm performance due to overlap. For example, the sale of shares to comply with B-BBEE could indicate social responsibility resulting in favourable share price reaction, it could imply that a firm would become eligible for government contracts and therefore favourable share price reaction, or it could imply a corporate reorganization which would eliminate unproductive employees and therefore create a favourable share price reaction. Therefore, rather than focusing on the driving cause of the relationship, it is more appropriate to focus on the nature of the relationship between B-BBEE policy and firm performance.
\section{Previous research}
Indeed previous research also focus on the nature of the relationship and avoid specific drivers of the relationship between B-BBEE policy and firm performance (\citeauthor{N7}, \citeyear{N7}, p549; \citeauthor{N27}, \citeyear{N27}, p86; \citeauthor{N4}, \citeyear{N4}, p45) . Exemplary for this is the study by Merwe and Ferreira (\citeyear{N7}, p549) wherein they state that B-BBEE compliance involves cost and benefits relating to social responsibility and market access, but the authors merely state that when benefits exceed costs, then a positive relationship exists between B-BBEE compliance and firm performance, avoiding whether costs and benefits to either social responsibility or market access drive the relationship.

Merwe and Ferreira test the effect of B-BBEE aggregate score on share price return. Merwe and Ferreira (\citeyear{N7}, p550) selected thoroughly tested control variables that impact share price return, complimented by a industry dummy vector. These control variables were sourced from the famous three factor Fama and French model.This widely adopted three factor model of Fama and French captures most of the average share price return (\citeauthor{N52}, \citeyear{N52}, p1998; \citeauthor{N53}, \citeyear{N53}, p39 ; \citeauthor{N7}, \citeyear{N7}, p549) . The Fama and French model state that share price return is a function of risk free rate, sensitivity of a firm’s share to market risk premium, size and value. The market risk premium is based upon the traditional CAPM model. This factor essentially captures the co-movement of a firm to market movement. The size factor relates to the market capitalization of a firm. This is simply the market price of a firm on a stock exchange multiplied by the shares outstanding of the firm. The larger the market capitalization the larger the size of a firm. Fama and French find that smaller firms tend to outperform larger firms \cite[p38]{N53}. The value factor relates to the book to market ratio of a firm. The book to market ratio is equal to the book value (also known as the shareholders equity value) as stated on the balance sheet, divided by the market capitalization of the firm. Fama and French state that the market undervalues distressed, high book to market ratio firms, and therefore these firms tend to outperform low book to market ratio firms \cite[p1975]{N52} . Merwe and Ferreira (\citeyear{N7}, p550) omit the market risk premium and add the earnings to price ratio. It is not clear why Merwe and Ferreira omit the market risk premium variable. However, Fama and French do find that the earnings to price ratio holds explanatory value for share price return \cite[p1997]{N52}. After establishing a model with the appropriate control variables Merwe and Ferreira examine their model from 2005 to 2011, capturing 905 observations \cite[p550-551]{N7}. The B-BBEE aggregate scores were retrieved from Empowerdex \cite[p550]{N7}. Merwe and Ferreira (\citeyear{N7}, p550) note that the Empowerdex data is released annually in April. To incorporate time for the market to incorporate this information, Merwe and Ferreira measure share price return from August to August. Using these specifications, Merwe and Ferreira find a significant negative relationship between B-BBEE aggregate score and share price return [7, p552]. However, Merwe and Ferreira note that their study could be subject to a biased due to the time period as Codes of good practise were revised in 2013 and suggest that research should be done towards the long term effects of B-BBEE compliance as Merwe and Ferreira (\citeyear{N7}, p555) only using one year forward share price returns.

Mehta and Ward study the long term effect of the B-BBEE aggregate score on share price return in a different manner. Rather than performing a regression analysis, Mehta and Ward (\citeyear{N27}, p90) compose 4 portfolios, based on their rank. Put straightforward, the top portfolio consists of the top 25\% B-BBEE scoring firms. Each quarter the portfolios were rebalanced to account for changes in B-BBEE score as well as changes in the sample size (\citeyear{N27}, p90). Mehta and Ward then create a range of normality by bootstrapping which creates a top and bottom range of average share price return for the entire sample. The share price return of the top 25\% B-BBEE is then plotted against this range of normality, if the share price returns exceed the top line of the range of normality then the relationship between B-BBEE policy and firm performance would be deemed significantly positive. The time period Mehta and Ward covered was 2009 - 2015, covering 160 firms \cite[p89, p94]{N27}. B-BBEE aggregate scores were sourced directly, or through Mpowered Business Solution \cite[p89]{N27}. Mehta and Ward (\citeyear{N27}, p95) find a negative relationship between B-BBEE aggregate score and share price return. The portfolio with the top 25\% B-BBEE scoring firms, performed the worst. However, what is noteworthy is that Mehta and Ward did not create the portfolios industry neutral. This means that the portfolio with the top 25\% B-BBEE scoring firms could consist of firms all operating in a industry facing share price return decline. By not controlling for industry, Mehta and Ward were vulnerable to capturing industry effects as well as the effect of B-BBEE on share price return.

Mokgobinyane (\citeyear{N4}, p46) uses the B-BBEE aggregate score as the treatment variable, with control variables size, leverage, liquidity and industry. These variables were then tested with different dependent variables, Revenue, Net profit margin and return on equity \cite[p48-49]{N24}. These three regression models are then tested separately for three years, namely 2007, 2010 and 2013 \cite[p53]{N24}.  Mokgobinyane (\citeyear{N4}, p52)  uses the B-BBEE aggregate score of the year prior (i.e. the B-BBEE aggregate score of 2006 for the 2007 analysis) to measure whether the score impacted the dependent variable in the subsequent year to measure the causal effect of the B-BBEE aggregate score. Furthermore, Mokgobinyane (\citeyear{N4}, p57)  compares his sample with B-BBEE aggregate scores, sampled from the B-BBEE aggregate score of the top 100 B-BBEE scoring firms published by Empowerdex, to a sample group of firms listed on the Johannesburg Stock Exchange that are not included in the top 100. Mokgobinyane (\citeyear{N4}, p3) finds a disperse set of values for the B-BBEE score coefficients for the various models, none of which displaying a significant relationship between B-BBEE score and the different firm performance measurements. Data availability limits the ability to generalize the findings of Mokgobinyane. Using only three specific years, the study makes it self vulnerable to selection bias. This is underlined by the fact that the new Code of Good Practise released in 2013 was not included in Mokgobinyane’s research. Furthermore, Mokgobinyane industry control variables were different for the different years. For example for 2010 and 2013 industry classifications Basic materials, Consumer Services, Financial and Industrial were used. For 2007, however, Technical, Industrial, Financial, Basic materials, Consumer services, consumer goods and health care industry classification were used. This could indicate sample size bias, i.e. the 2007 dataset was more dispersed compared to the 2010 and 2013 dataset. Mokgobinyane’s selection for revenue as a measurement of firm performance is debatable, as mentioned the measures Mokgobinyane used are susceptible to accounting gimmicks.

Acemoglu et al. (\citeyear{N23}, p29) test the impact of B-BBEE aggregate score on profitability as well, using net profit margin as the dependent variable. By conceptualizing the relationship between B-BBEE aggregate score and profitability, Acemoglu et al. simplify the selection of control variables. Acemoglu et al. (\citeyear{N23}, p26) argue that the efficacy of B-BBEE aggregate score depends on the cost and benefits of a firm on being B-BBEE compliant. Therefore, control variables include fraction of shares held by the South African Public Investment Corporation and industry with industry charters \cite[p27]{N23}. Acemoglu et al. (\citeyear{N23}, p29), similar to Mokgobinyane, relies on Empowerdex for B-BBEE aggregate score. The Acemoglu et al. (\citeyear{N23}, p29) paper tests the impact of B-BBEE score on net profit margin for the period 2004-2006, yielding 159 observations . Acemoglu et al. (\citeyear{N23}, p32) find positive, but not significant, relationship between B-BBEE aggregate score and net profit margin]. The sample size is rather small, focussing only on the 2004-2006 time period. This makes it difficult to generalize this study’s finding on the relationship between B-BBEE aggregate score and net profit margin. Also, although intuitively it makes sense to focus on drivers of B-BBEE cost and benefits to select control variables to prevent omitted variable bias, even this approach is not devoid of omitted variable bias. Considering the impact that the foreign funding stop had on the Apartheid regime, perhaps a wider definition should have been adopted, rather than only controlling for shares held by one institutional investor, in this paper the South African Public Investment Commission. 
\subsection{Summary and conclusion}
This study evaluates previous research on the relationship between B-BBEE policy and firm performance. The Merwe and Ferreira model used a proven set of control variables based on the Fama and French model, whereas other studies either used accounting measures of lacked control variables to reach a general conclusion on the relationship between B-BBEE policy and firm performance. This study finds that research hints towards a negative relationship, therefore the more a firm complied to B-BBEE the lower their firm performance. Previous research was found limited by time horizon, measuring only specific years and measuring the effect of B-BBEE policy on a one year basis. The aim of the policy is long term, therefore it is more appropriate to measure the relationship of B-BBEE policy on firm performance using time horizons longer than a year. Previous studies on focusing on specific years, for example 2005 - 2011, are vulnerable to bias to this particular set of years.
\section{Summary and hypotheses}
To answer the research question “What is the long term relationship between B-BBEE policy and firm performance?”, a theoretical framework is required to provide a comprehensive answer. This chapter initiated the theoretical framework by  discussing firm performance. A firm’s responsibility extends past mere profit generation. However, all responsibilities, profit generation as well as being a social responsible entity, is captured through the profitability of a firm. Therefore this study equates firm performance to profitability of firms. Profitability is best operationalized through share price return.

Profitability of firms, as a reflection of the wide spectrum of responsibilities, touches upon the effects that B-BBEE policy compliance has on a firm. B-BBEE policy affects profitability negatively through costs incurred to comply to B-BBEE and beneficially through either efficiency gains or revenue expansion related to B-BBEE. In other words, B-BBEE policy is related to firm performance because B-BEE policy involve cost and benefits which affect firm performance. For example, the costs of B-BBEE policy in terms of ownership targets deal with shares of the firms sold to Black people at discount to market prices, diluting shareholders and lowering share price return. Alternatively, B-BBEE policy requires other expenditures as well, such as skill development. This chapter did not identify which specific cost or benefit dominates the relationship between B-BBEE policy and firm performance. Due to overlap of specific the costs and benefits, this study, as well as most previous research, prefers to focus on the aggregate of the cost and benefit associated with B-BBEE policy compliance on firm performance. However, analysis of the benefits of B-BBEE policy does indicate that firms operating in sectors interacting with government could enhance revenue as government entities must select B-BBEE compliant business partners. This leads to the indication that the aggregate effect of B-BBEE policy compliance differs across sectors. Therefore to provide a comprehensive answer to the research question, one must explore whether the long term relationship between B-BBEE policy and firm performance varies per sector. The following (falsifiable) hypothesis is formulated:
\begin{nullhypothesis}The relationship between B-BBEE policy and firm performance is uniform across sectors\end{nullhypothesis}

Moreover, the Contextualization chapter revealed an increasingly stringent B-BBEE policy. This indicates that there is not only a cross sectional dimension to the relationship between B-BBEE policy and firm performance, but also a time dimension. Specifically, because the policy became more interventionist it is expected that the relationship between B-BBEE policy and firm performance became more pronounced. The following (falsifiable) hypothesis is formulated:
\begin{nullhypothesis}The intensity of the relationship between B-BBEE policy and firm performance is uniform through time\end{nullhypothesis}

The apprehensive strategy firms adopted towards Black Economic Empowerment hints towards a negative relationship B-BBEE policy between and firm performance in general. The majority of previous research utilized linear regressions to investigate the relationship between B-BBEE policy and firm performance.  The lionshare of research found negative relationship, indicating that the aggregate of cost and benefits of B-BBEE policy compliance detracted from firm performance. These studies investigated particular years in time. For example, Merwe and Ferreira examine their model from 2005 to 2011, Mokgobinyane used 2007, 2010 and 2013 and Acemoglu et al. tested the relationship between 2004 and 2006, and Mehta and Ward tested 2009 to 2015 (\citeauthor{N7}, \citeyear{N7}, p550; \citeauthor{N4}, \citeyear{N4}, p53 ; \citeauthor{N23}, \citeyear{N23}, p29; \citeauthor{N27}, \citeyear{N27}, p94). With the exception of Acemoglu et al., all research indicated a negative relationship between B-BBEE policy and firm performance. This leads to the indication that on overall, the relationship between B-BBEE policy and firm performance would be negative. The following (falsifiable) hypothesis is formulated:
\begin{nullhypothesis}The relationship between B-BBEE policy and firm performance between 2004 and 2018 was positive\end{nullhypothesis}
