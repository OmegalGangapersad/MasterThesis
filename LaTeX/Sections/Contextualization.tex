This chapter introduces BEE to the reader. Historical background, which will add relevant color to the theory will be outlined. Furthermore, this chapter shares the mechanics of B-BBEE to the reader which will add value in terms of understanding the empirical results and the increasingly interventionist stance the ANC government has taken to tackle wealth inequality.
\section{The history of BEE and B-BBEE}
The main focus of this section is to understand the persistent wealth inequality in South Africa from a historical perspective. This section is divided in three subsections. The first subsection discusses the Apartheid era to outline how wealth inequality was not just a negative consequence but a feature of Apartheid. The second subsection outlines BEE, to inform the reader that BEE was never planned meticulously by the ANC, but rather loosely. This is done by addressing the Freedom Charter in 1955 and the informal arrangement just after Apartheid. In this section the reader will also learn how this loose policy resembles the broader ESG scope, in which in the early days governments all over the globe did not strictly enforce ESG, but rather opted to let the industry self-regulate. The third and final section describes the B-BBEE as a set of formal rules (which will be also described in this section). This is juxtaposed against other ESG measure in analyzed to see why there is still criticism on B-BBEE.
\subsection{Apartheid era}
BEE was installed to address the persisting wealth inequality mainly between the disenfranchised Black people and the historically disproportionally advantaged White people. This imbalance originated from the Apartheid era. During this era, Black South Africans were restricted from the freedom to politically and financially participate within society [24, p3]. Instead, wealth and political power were initially controlled by the minority White South Africans [4, p5].  Black people represented less than 3\% of managerial positions in 1990, and 99\% of equity ownership was held by White people in 1995 [24, p5]. Dr. Motsuenyane, post Apartheid Supreme Chamber of Commerce President, stated: “We were restricted from trading also in such a way that we could present competition to White traders” [30, 104]. Secondly, the Apartheid system cultivated a system in which Black people were mostly limited to perform unskilled labor. The Bantu Education Act of 1953 lowered the standard of education of Black people compared to White people, ultimately resulting in a disparity in per capita income. In 1970 of per capita income of Black people was 3,133 South African Rand (hereafter, ZAR) and 45,751 ZAR for White people [30, 104-5].

The international society condemned these unjustifiable policies by blocking South African (White) companies from entering the international markets [24. p3]. International ties were broken and firms exited the South African market [24, p3]. It could be argued that the exiting of international competition from the South African market, albeit temporarily, strengthened White owned South African companies as competitive forces diminished. However, most scholars indicate that the international pressure caused the downfall of the Apartheid regime. Confirming that the Apartheid regime shielded their economical interests from both external competition and internal competition, Lowenberg notes that the Apartheid regime installed import substitution policies in which Black people were barred, as mentioned earlier as well, to engage in enterprising [31, p65]. Lowenberg continues by stating that import substitution policies cause imbalance in capital accounts, thereby leaving a deficit in balance of payment [31, 65]. Therefore, the benefits of isolation of the South African economy to the White companies, were short lived. In 1985, resulting from anti-Apartheid civil pressure within the United States, Chase Manhattan (later merged into JP Morgan Chase, one of the largest banks in the United States) stopped providing short term funding (loans) to South Africa, which triggered a wholesale funding stop from international financial institutions to the Apartheid regime [32, p324].
\subsection{The first phase of BEE, BEE as the transfer of shares}
The South African private sector, feeling the brunt of the international boycott, began to cater to the international demands. The initiatives these enterprises undertook could be seen as the first form of Black Economic Empowerment [23, p6]. For example, in 1993 Sanlam a White owned financial services firm sold 10\%  of its shares to well connected Black politicians [23, p6]. Contrary, Lowenberg notes that even in the 1970s, firms constrained by Apartheid legislation constraining labor supply (for work only for White people), circumvented some of these exclusionary laws. Firms for example renamed jobs so that these firms could tap into the vast supply of Black labor [31, p63]. Randall further underlines the finding that Black Economic Empowerment initiated from the agility of White private companies, not only pressured by the international community, but also anticipating a regime switch (to a post Apartheid era) wherein White companies had to provide access to the markets to Black people [6, p14].

The Freedom Charter in 1955, which united the Anti-Apartheid movement under the banner of the ANC, called for economic freedom for all South Africans. It is worth noting that the South African Communist Party also assimilated under the banner of a united anti-Apartheid movement led by the ANC [33, p16]. This could indicate the eagerness of the White private sector to be viewed as cooperating. The incoming regime, the ANC regime, was at minimum heavily influenced by anti-capitalistic philosophies and firms felt the need to cater to some of the demands of this new power to survive. Nelson Mandela, when elected in 1994, indeed prioritized wealth redistribution to correct for Apartheid engineered wealth inequality [24, p5]. This required developing strategic goals, these goals consists of underlying concepts. The paper by Tshetu first describes the concept of empowerment which pertains to correcting historical wrongdoings towards disadvantaged majorities and re-allocating social, economic and political power to the previously disadvantaged [6, p11]. Economic empowerment is a concept that is framed around empowerment alluding to positive outcomes in the economy. Essentially economic empowerment encompasses citizens equal access into the economic stream of the country, whereby disadvantaged communities and the vulnerable are able to access assets and economic opportunities [6, p11-12]. In the South African context of addressing the past mistakes of failed markets, Black Economic Empowerment (BEE) had become the momentous focal [4, p5].

Tshetu identifies and discusses three policies associated with the ANC and BEE. The first being an economic policy that was mapped out by ANC’s Macro-Economic Research Group in 1991, the radical ‘Ready to Govern’ policy was the second economic policy which referred to an economic turning point for opening up the economy [6, p13; 25, p525]. The final policy was Reconstruction and Development Programme (RDP) that focused on economic growth for disadvantaged majorities [6, p14]. However, none of these policies guided private sectors to close the wealth inequality gap. Fruend assigns the poorly directed policies as a result of lack of policies of the ANC, but abundance of philosophies prior to their assumption of power [25, p520].  Despite these initiatives, the lack of focussed policy efforts kept BEE mainly in control of the White private sector, with minimal government control, in this first phase of BEE [23, p6]. BEE mostly manifested in BEE transactions; the sale of shares of White companies to the Black elite. An estimated 231 transfer deals were closed until 1998, resulting in a significant growth from 1\% of capital ownership in 1995 by Black people to an increase of 10\% ownership in 1998 [23, p5-p6]. A Black elite was created which had interests aligned with White companies, and the establishment of which, by the White companies, created a positive image for White companies [27, p86]. President Ramaphosa, having played a crucial role in the ANC road to victory and labor union leader entered the private sector arena in 1996 which resulted in the establishment of his company Sanduka which acquired shares in Lonmin [30, p98]. Consider a fictitious firm called Shopwrong. Typically, BEE transactions were structured as follows; capital deficient Black influential people (like union leaders or ANC politicians) borrowed from White owned public firm called Shopwrong to finance their purchase of the shares, at a 15-40\% discount to prevailing market price of Shopwrong [23, p5]. In order to service the interest payments, the Black influential people used dividends from Shopwrong. 
\subsection{The second phase of BEE, B-BBEE as the formalization of BEE policy}
Most scholars uniformly agree that the BEE transactions and the three policies associated with BEE did not eliminate wealth inequality. Lindsay notes that BEE, from a political perspective, had become a “slippery catch phrase” for various ideological persuasions for politicians [30, 3]. Tsehtu, from the perspective of the White corporations, concludes that the corporate sector did not pursue BEE in all earnest, but from a self preservative motivation. Evidence of this was that firms unbundled key assets and only sold non-key assets to the Black elite [6, 15]. Mokgobinyane observes that even the non-key asset ownership of the Black elite, reduced over time from 10\% to 4.3\% in 1998 [4, p18]. Forced selling, as the Asian Crisis in 1998 wrecked havoc in the global markets and eroded profits of South African firms, reduced dividend payouts. These dividends, as mentioned in the previous paragraph on the BEE transaction structure, were crucial to service the interest payments which financed the BEE transactions for the Black elite. However, even prior to 1998 government realized that the informal corporate led BEE policy did not fundamentally alter distribution of wealth, rather wealth remained not inclusive for all South Africans [24, p8].

The persisting wealth inequalities introduced the second phase of BEE, formalizing BEE policies [23, p7]. The ANC government established the BEE Commission in 1998 headed by president Ramaphosa [23, p7]. The BEE Commission’s responsibility was to recommend effective policy and measures, which would reduce wealth inequality broadly (all Black citizens) [6, p15]. The BEE Commission recommended for increased state intervention to promote Black empowerment [23, p7]. As part of  the BEE Commission’s responsibility of assessing BEE effects, the BEE Commission concluded that the proportion of shares owned by Black people was not a good way to measure BEE’s effectiveness [6, p18]. Rather, in light of the leveraged purchase of ownership that backfired during the Asian Crisis, in 2001 the BEE Commission stated that ownership would only be considered Black ownership if debt that financed the purchase of the ownership was fully serviced [23, p8]. The BEE Commission recommended explicit guidelines. A Black firm was a firm where more than 50.1\% was owned and managed by Black people [23, p8]. A Black empowered firm was a firm were more than 25.1\% was owned and managed by Black people [23, p8]. In 2003, the B-BBEE Act 53 came into effect. This Act present broad objectives which encompassed a vision for Black empowerment. This Act introduced the Code of Good Practise which a set of specific guidelines intended for firm to support Black empowerment. The Act also established the BEE Advisory Council [37, p7]. The BEE Advisory Council would make recommendations to the South African cabinet on specific targets to be set in the Code of Good Practise. Industry groups started negotiations with the BEE Advisory Council, after the BEE Commission drafted a recommendation to force industries to transform themselves into Black companies within 10 years [23, p9]. The result of these negotiations was the strategy for Broad-Based Black Economic Empowerment document, which outlined Codes of Good Practice in 2007 [23, p9; 6, p16]. These Codes translated B-BBEE into a B-BBEE scorecard, a balanced scorecard consisting of seven elements which will be further discussed in the section “The mechanics of the B-BBEE score” below.[23, p9]. The Codes act as the baseline for measuring a firm’s B-BBEE rating and compliance [23, p9]. Government entities committed themselves to at least consider B-BBEE scores in decision making with private sector regarding  licensing, procurement, public-private partnership and the sale of state-owned entities [35, p8]. The BEE Commission stated that this commitment should have consequences for the entire formal economy [35, 9]. Government entities selected suppliers with a good B-BBEE score and suppliers in turn would require their business partners to have a good B-BBEE score because this would further boost B-BBEE score of the suppliers [29, p5].  As such, B-BBEE score posed a significant intervention into the marketplace. In 2013 the Codes and scorecard were amended [6, p16].  The amendments posed increased pressure on specifically large companies to engage in Black empowerment. Compliance target for management control were amended such that companies were subject to annually set targets, and increased targets for skill and development. More importantly, public entities were now obliged to apply B-BBEE targets, rather than just taking into consideration B-BBEE scores when selecting suppliers [4, p36]. The amendments also targeted “fronting”, or the placement of Black people in management position without any mandate and the sole goal to broadcast adherence to B-BBEE norms [4, p36]. Finally the number of elements comprising the B-BBEE aggregate score was trimmed from seven to five [36, p8; 36, p15]. These amendments suggest that government intervention to force Black empowerment increased. However, Acemoglu projects, that firms should evolve as Black empowerment is broadly realised [23, p11]. This then, should allow government to retrace current interventionist measures.
\section{The mechanics of the B-BBEE scorecard}
The Department of Trade and Industry through the BEE Commission assumed responsibility of overseeing compliance B-BBEE scores, which included but not limited to B-BBEE (of share transfers) transactions, supervise compliance to the B-BBEE Act [36, p4]. Independent BEE verification agents, accredited by Department of Trade Department SANAS (South African National Accreditation System) audit companies to give them B-BBEE scores [38, p5]. The B-BBEE is comprised of various elements, with varying weight. The independent BEE verifier scores a company on each element. 
\subsection{Mechanics 2003 to 2007}
From its onset in 2003 until 2007 the B-BBEE scores were moreso a measurement based upon market participants interpretation. The B-BBEE Act 2003 laid out seven objectives to achieve Broad Based Black Economic Empowerment, namely:
\begin{enumerate}
  \item Promoting economic transformation to provide access to Black people to participate in the economy
  \item Achieving a substantial change in the racial composition of ownership and management 
  \item Increase the extent to which communities, workers, cooperative and other collective enterprises own and manage existing and new enterprises and increasing their access to economic activities, infrastructure and skills training
  \item Increase the extent to which Black women own and manage existing and new enterprises, and increasing their access to economic activities, infrastructure and skills training
  \item Promote investment programs that lead to broad-based and meaningful participation in the economy by Black people in order to achieve sustainable development and general prosperity
  \item Empower rural and local communities by enabling access to economic activities, land, infrastructure, ownership and skills
  \item Promoting access to finance for Black economic empowerment [37, p5; 37, p6]
\end{enumerate}
\subsection{Mechanics 2007 to 2013}
In 2007, the first Codes of good practise were released which further guided the market place how to gauge their extent of  participation into B-BBEE. The below table displays the framework.
\begin{table}[H] %H forces the position of the table at the line where you place it (not on a separate page etc) -https://tex.stackexchange.com/questions/121155/how-to-adjust-a-table-to-fit-on-page https://tex.stackexchange.com/questions/332528/increasing-the-space-between-two-rows?rq=1
\centering
\caption{B-BBEE Codes of good practise 2007} 
\resizebox{\textwidth}{!}{\begin{tabular}{lll}

  \bottomrule
  \\
 Element                    & Weight & Most Notable Targets   \\ \\
  \midrule
Ownership                  & 20\%   & 25\% of company's shares owned by Black people, 10\% of company shares owned by Black women         \\
Management Control         & 10\%   & 40\% of management structures should be Black people                                                \\
Employment Equity          & 15\%   & Employ a majority of Black people in various roles and positions                                    \\
Skills Development         & 15\%   & At least 3\% of total payroll spend on developing skills of Black employees                         \\
Preferential Procurement   & 20\%   & Buy at least most of the raw materials and other products and services from BEE-compliant companies \\
Enterprise Development     & 15\%   & Encourage companies to invest in developing small businesses that are Black-owned                   \\
Socio-Economic Development & 5\%    & Spend at least 1\% of profits on socio-economic programmes and organisations on Black beneficiaries
      \\ 
   \bottomrule
\end{tabular}}
\end{table} 
These above mentioned target were set for the Generic codes which sets targets for unspecified industry. For specific industries such as the Mining Industry different targets were set. Furthermore, the Codes differentiated between the size of companies in three categories: Generic Enterprises, Qualifying Small Enterprises and Exempted micro-enterprises [4, p38]. The differentiation distincts the extent to which a company should comply to all the codes to be B-BBEE verified. It is important to take note that governmental and public entities are obliged to be B-BBEE compliant, however for private firms it is not obligatory rather voluntary and there direct and indirect incentives that may promote compliance [42 p682].  Companies with revenue of more than 35 million should comply to all the B-BBEE codes, according to the 2007 Codes of good practise. Each element has a corresponding weight in the total score. Therefore the total B-BBEE score is a weighted average score. Based on the score, the amount spent under preferential procurement could be multiplied by a percentage. For example, consider a business to business transaction of 100 South African Rand wherein the selling company with 100 points B-BBEE score would enable the buying company to claim 135\% of the 100 South African Rand as amount spend on B-BBEE compliant companies. This demonstrates that the B-BBEE score should, as government envisioned, marketplace wide implications (as mentioned in the section the second phase of BEE, B-BBEE as the formalization of BEE policy).
\subsection{Mechanics 2013 onwards}
The 2013 amended Codes of good practise tightened targets and obliged public entities to incorporate B-BBEE score into their decision making for conducting business, rather than just consider the B-BBEE score of private firms. Government, in fact, would give preference to firms that are B-BBEE compliant [46, p9]. A firm is B-BBEE compliant when it complies to all the targets of B-BBEE and above level 8 as a contributor [46 p9]. See below the amended codes and B-BBEE targets since 2013.
\begin{table}[H] %H forces the position of the table at the line where you place it (not on a separate page etc) -https://tex.stackexchange.com/questions/121155/how-to-adjust-a-table-to-fit-on-page https://tex.stackexchange.com/questions/332528/increasing-the-space-between-two-rows?rq=1
\centering
\caption{B-BBEE Codes of good practise 2007} 
\resizebox{\textwidth}{!}{\begin{tabular}{lll}

  \bottomrule
  \\
 Element                    & Weight & Most Notable Targets   \\ \\
  \midrule
Ownership                                      & 25\%   & 25\% of company's shares owned by Black people                                                                                     \\
Management Control                             & 15\%   & 50\% of management structures should be Black people                                                                               \\
Skills Development                             & 20\%   & At least 6\% of total payroll spend on developing skills of Black employees                                                        \\
Enterprise Development \& Supplier Development & 40\%   & 25\% of cost of sales ex. labor costs and depreciation must be spent in South Africa. 50\% of job created must be for Black people \\
Socio-Economic Development                     & 5\%    & Spend at least 1\% of profits on socio-economic programmes and organisations on Black beneficiaries                               
      \\ 
   \bottomrule
\end{tabular}}
\end{table} 
The changes in the respective weights display the increasing realisation to emphasize the internal market. This relates to the Enterprise Development & Supplier Development, which was assigned the largest weight in the 2013 Codes, 40\%, whereas the comparable Preferential Procurement in the 2007 Codes was assigned a 20\% weight in the total B-BBEE score. This indicates that the measure to force Broad Based Black Economic Empowerment to tackle wealth inequality was seen best tackled by forcing private and public sector to interact with companies that also actively engage in Broad Based Black Economic Empowerment.
\section{Summary and conclusion}
This section has brought together the history and mechanics of BEE and B-BBEE. The context of this thesis interchangeably refers to BEE or B-BBEE because of the historical time frames. Although it may interchange, BEE would refer to the period before and after Apartheid and changes in the year 2000 to  B-BBEE which would refer to the period after Apartheid in 2003. The Apartheid era which ended in 1991, had caused chaos for the South African economy. During this time there was divide between Black and White people, defining who could and could not participate socially, politically and economically.

The international society did not favor this divide and as a result blocked South Africa from international participation, imposed sanctions and stop providing funds to South Africa. With a crippling economy South African private firms took initiatives to bring change and empower Black people. Private firms had began transactions to sell their shares to Black People at discount rates. This was regarded to be the first phase of empowering Black people. However, scholar argue that the initiatives were of self interest and not related to the international issue. It was the fall of Apartheid in 1991 that allowed Black people to gain political power however, the they lacked economic power. The ANC had taken political power had envisioned BEE. They had created policies to address Black empowerment however, these policies did not guide private sectors to empower all Black people. BEE initially in the hands of firm translated to the transfer of shares which only benefitted a few and created a Black elite.

Wealth inequality persisted forcing government to increase intervention which led to the second phase of BEE. In 1998 BEE became an official policy and the ANC government had setup the BEE Commission. The BEE Commission was responsible for overseeing recommendations and compliance of BEE. Recommendations included installed measures to monitor BEE compliance. These measures translated to the BEE scorecard that was introduced in 2003. During 2003 to 2007 this scorecard consisted of seven elements that would economically empower Black people, however the scores were only measuring and depicting market participation. The Code of good practice was implemented in 2007 to further guide firms to comply with BEE. These Codes were amended in 2013 and as a result the BEE scorecard also changed. The change meant strict regulation for public entities. Compliance to B-BBEE was obligatory for the selection of suppliers.

The history of BEE policy and the B-BBEE scorecard amendments reflect that ineffectiveness drove the increasingly interventionist stance of the South African government. These observations of ineffectiveness and increasing interventions could indicate low commitment of firms to commit to the empowerment of Black people.