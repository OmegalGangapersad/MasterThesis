The long term relationship between B-BBEE policy and firm performance requires understanding on the evolution of the B-BBEE policy through time. This chapter presents the reader with that evolution, which uncovers the dynamics between South African government's and South African firms as it pertains to ultimate goal of B-BBEE policy, Black Economic Empowerment.
\section{The evolution of B-BBEE policy}
The dynamics between the South African government and firms on Black Economic Empowerment are deep rooted, out dating the installation of the B-BBEE Act in 2003. Therefore this section is divided into the pre B-BBEE policy era and the B-BBEE policy era.
\subsection{Pre B-BBEE policy era}
The B-BBEE policy evolved from BEE and restrictions from Apartheid. During Apartheid era, the Bantu Education Act of 1953 lowered the standard of education of Black people compared to White people, ultimately resulting in a disparity in per capita income. In 1970 of per capita income of Black people was 3,133 South African Rand (hereafter, ZAR) and 45,751 ZAR for White people \cite[p104-p105]{N30}. 

The international society condemned the unjustifiable policies of South African firms toward Black people by blocking South African (White) companies from entering the international capital markets
\cite[p3]{N24}. International firms also exited the South African market \cite[p3]{N24}. In 1985, resulting from anti-Apartheid civil pressure within the United States, Chase Manhattan (later merged into JP Morgan Chase, one of the largest banks in the United States) stopped providing short term funding (loans) to South African firms, which triggered a wholesale funding stop from international financial institutions to the South African firms \cite[p324]{N32}. South African firms, feeling the brunt of the international boycott, had no option but to adjust and began to cater to the international demands. In 1993 Sanlam a White owned financial services firm sold 10\%  of its shares to well connected Black politicians \cite[p6]{N23}. The initiatives could be seen as the first form of Black Economic Empowerment \cite[p6]{N23}.

The African National Congress (ANC) did not prepare targeted policies for firms to empower Black people when it assumed power, therefore firms remained in power to control their implementation of BEE \cite[p131]{N30}. Led by South African firms, BEE mostly manifested in BEE transactions. The transaction entail the sale of shares of White companies to the Black elite. The mechanics of the sale of shares reveal that firms used sharetransfer to prolongate their position. To describe these mechanics, consider a fictitious firm called Shopwrong. Typically, BEE transactions were structured as follows; capital deficient Black influential people (like union leaders or ANC politicians) borrowed from White owned public firm called Shopwrong to finance their purchase of the shares, at a 15-40\% discount to prevailing market price of Shopwrong \cite[p5]{N23}. In order to service the interest payments, the Black influential people used dividends from Shopwrong. An estimated 231 transfer deals were closed until 1998, resulting in a significant growth from 1\% of capital ownership in 1995 by Black people to an increase of 10\% ownership in 1998 \cite[p5-p6]{N23}. A Black elite was created which had interests aligned with White companies, and the establishment of this elite, by the White companies, created a positive image for White companies \cite[p86]{N27}. Most scholars uniformly agree that the BEE transactions did not eliminate wealth inequality. Lindsay (\citeyear{N30}, p3) notes that BEE, from a political perspective, had become a “slippery catch phrase” for various ideological persuasions for politicians. Tsehtu (\citeyear{N6}, p15), from the perspective of the White corporations, concludes that the corporate sector did not pursue BEE in all earnest, but from a self preservative motivation,  evidenced by White corporations only selling non-key assets to the Black elite. Mokgobinyane (\citeyear{N4}, p18) observes that even the non-key asset ownership of the Black elite, reduced over time from 10\% to 4.3\% in 1998. Forced selling, as the Asian Crisis in 1998 wrecked havoc in the global markets and eroded profits of South African firms and reduced dividend payouts. These dividends, as mentioned earlier, were crucial to service the interest payments which financed the BEE transactions for the Black elite. However, even prior to 1998 government realized that the informal corporate sector led BEE policy did not fundamentally alter distribution of wealth, rather wealth remained not inclusive for all South Africans \cite[p8]{N24}. 
\subsection{B-BBEE policy era}
The persisting wealth inequalities heralded the second phase of BEE, which is the time period this study investigates, the phase in which formalized BEE policies into B-BBEE \cite[p7]{N23}. In 2003, the B-BBEE Act 53 came into effect. This Act presented broad objectives which encompassed a vision for Black empowerment, exceeding the mere transfer of equity ownership. Exemplary of government’s intention for the B-BBEE Act, the advisory organ for BEE, the BEE Commission called for an “unapologetic and interventionist” policy \cite[p168]{N30}. The Act also established the BEE Advisory Council \cite[p7]{N37}. The BEE Advisory Council made recommendations to the South African cabinet on specific targets to be set in the Code of Good Practise. The Code of Good practise was a document which included specific targets for firms to comply to, to be compliant to the B-BBEE policy. Anticipating government intervention, firms, just as Sanlam in 1993, started self imposing targets prior to the recommendations \cite[p9]{N23} . This placed them in position to negotiate with the BEE Advisory Council \cite[p9]{N23}. Together the set of specific targets from the Codes of Good Practise were bundled into a balanced scorecard, called the B-BBEE scorecard. The higher the aggregate B-BBEE score, the more a company complied to the B-BBEE policy. Below an overview of the 2004 B-BBEE scorecard.
\begin{table}[H] %H forces the position of the table at the line where you place it (not on a separate page etc) -https://tex.stackexchange.com/questions/121155/how-to-adjust-a-table-to-fit-on-page https://tex.stackexchange.com/questions/332528/increasing-the-space-between-two-rows?rq=1
\centering
\caption{B-BBEE Codes of good practise 2007} 
\resizebox{\textwidth}{!}{\begin{tabular}{lll}
  \bottomrule
  \\
 Element                    & Weight & Most Notable Targets   \\ \\
  \midrule
Ownership                  & 20\%   & 25\% of company's shares owned by Black people \\
& & , 10\% of company shares owned by Black women         \\
Management Control         & 10\%   & 40\% of management structures should be Black people                                                \\
Employment Equity          & 15\%   & Employ a majority of Black people \\ 
& & in various roles and positions                                    \\
Skills Development         & 15\%   & At least 3\% of total payroll \\
& & spend on developing skills of Black employees                         \\
Preferential Procurement   & 20\%   & Buy at least most of the \\
& & raw materials and other products \\
& & and services from BEE-compliant companies \\
Enterprise Development     & 15\%   & Encourage companies to invest \\
& & in developing small businesses that are Black-owned                   \\
Socio-Economic Development & 5\%    & Spend at least 1\% of profits \\
& & on socio-economic programmes and organisations on Black beneficiaries
      \\ 
   \bottomrule
\end{tabular}}
\end{table} 
The B-BBEE scorecard clearly shows the intention to shift away from ownership as “the” instrument for Black Economic Empowerment to elements such as Preferential Procurement, Employment Equity and Enterprise Development. The above mentioned target were set for the Generic codes which sets targets for unspecified industry. Recognizing importance of several industry, and after further negotiation with firms from several industries, specific targets were set for a specific industries. For example, the mining industry had to adhere to different targets for ownership. The draft version from the Council indicated 51\% Black ownership by 2010 for the mining sector, this panicked investors resulting in share price crashes of mining firms \cite[p9]{N23}. Reports commented on the possible impact of international funding, should such an intrusive target be imposed, drawing parallels to the suggested ownership target and similar policy actions in India, which led to the exit of Coca-Cola and IBM in India (\citeauthor{N62}, \citeyear{N62}, p5; \citeauthor{N6}, \citeyear{N6}, p22). Eventually the mining industry negotiated and targets were moderated to 26\% Black ownership in 2012, whereas the Financial industry committed itself to 10\% direct Black ownership by 2010 \cite[p9]{N23}. Regardless of the targets, private sector firms were not obliged to comply to B-BBEE policy \cite[p682]{N42}. In 2006, at least 20\% of firm still did not comply to B-BBEE, nor had any plans to do so, indicating apprehensiveness of firm to adopt the policy \cite[p23]{N6}.

The Codes of Good Practise of 2007, which included the B-BBEE scorecard, were “gazetted”, meaning officially linked to the B-BBEE Act in 2007 \cite[p16]{N6}. This Code of Good Practise identified sectors based targets, as well as the previous Generic codes for firms outside the sectors for which specific targets were set. Furthermore, the 2007 Code of Good Practise differentiated between the size of companies using three size categories: Generic Enterprises, Qualifying Small Enterprises and Exempted micro-enterprises \cite[p38]{N4}. The differentiation distincts the extent to which a company should comply to all the codes to be B-BBEE verified. It is important to take note that governmental and public entities were obliged to be B-BBEE compliant, however for private firms it was not obligatory rather voluntary \cite[p682]{N42}. Direct and indirect incentives were used to promote compliance for private firms \cite[p682]{N42}. These incentives centered around preferential procurement. Government and other public entities were to consider the B-BBEE compliance status of their suppliers. Therefore, B-BBEE compliance of firms could result in higher revenue through government project contracts.

The 2013 amended Code of Good Practise tightened targets and obliged public entities to incorporate B-BBEE score. Below the amended B-BBEE scorecard.
\begin{table}[H] %H forces the position of the table at the line where you place it (not on a separate page etc) -https://tex.stackexchange.com/questions/121155/how-to-adjust-a-table-to-fit-on-page https://tex.stackexchange.com/questions/332528/increasing-the-space-between-two-rows?rq=1
\centering
\caption{B-BBEE Codes of good practise 2013} 
\resizebox{\textwidth}{!}{\begin{tabular}{lll}

  \bottomrule
  \\
 Element                    & Weight & Most Notable Targets   \\ \\
  \midrule
Ownership                                      & 25\%   & 25\% of company's shares owned by Black people                                                                                     \\
Management Control                             & 15\%   & 50\% of management structures should be Black people                                                                               \\
Skills Development                             & 20\%   & At least 6\% of total payroll \\
& & spend on developing skills of Black employees                                                        \\
Enterprise Development \& Supplier Development & 40\%   & 25\% of cost of sales \\
& & ex. labor costs and depreciation must be spent \\
& & in South Africa. \\
& & 50\% of job created must be for Black people \\
Socio-Economic Development                     & 5\%    & Spend at least 1\% of profits \\
& & on socio-economic programmes and \\
& & organisations on Black beneficiaries                               
      \\ 
   \bottomrule
\end{tabular}}
\end{table} 
The changes in the respective weights display the increasing realisation to emphasize the internal market. This relates to the Enterprise Development & Supplier Development, which was assigned the largest weight in the 2013 Codes, 40\%, whereas the comparable Preferential Procurement in the 2007 Codes was assigned a 20\% weight in the total B-BBEE score. This indicates that the measure to force Broad Based Black Economic Empowerment to tackle wealth inequality was seen best tackled by forcing private and public sector to interact with firms that also actively engage in Broad Based Black Economic Empowerment. More importantly, public entities were now obliged to apply B-BBEE targets, rather than just taking into consideration B-BBEE scores when selecting suppliers \cite[p36]{N4}. The amendments also targeted “fronting”, or the placement of Black people in management position without any mandate and the sole goal to broadcast adherence to B-BBEE norms by criminalizing fronting \cite[p36]{N4}. Finally, the number of elements comprising the B-BBEE aggregate score was trimmed from seven to five \cite[p8,p15]{N36}. These amendments suggest that government intervention to force Black empowerment increased.
\section{Summary and conclusion}
The empowerment of Black people by firms appears to be mostly defensive strategy, with the aim to minimize impact of Black Economic Empowerment. In the pre B-BBEE policy era, international funding boycott, moved firms to reverse some of their exclusionary practises. Anticipating the end of Apartheid, firms such as Sanlam sought to appease the incoming ANC government by selling shares to Black people. This strategy, selling shares to a select, influential, group of Black people whilst retaining power continued well into the 2000s. The government responded by increasing intervention. The B-BBEE scorecard posed specific targets, expanding from the myopic sale of shares, to a broad based initiative that included amongst others  management control and supplier selection. As time progressed, the targets set in the B-BBEE scorecard, albeit after strong negotiation with firms, increased.

The strategy firms adopted towards Black Economic Empowerment hints towards a negative relationship B-BBEE policy between and firm performance. If Black empowerment were to increase firm performance, one could expect that firms would adopt a more progressive strategy. Government on the other hand, appeared to force firms into a more progressive stance, however remained cautious not to overstep itself and damage international relations. This further strengthens the earlier speculation that B-BBEE policy detracts firm performance. Regardless of the nature of the relationship, the increasing intervention of the B-BBEE policy should indicate an increasingly pronounced impact of B-BBEE policy on firm performance.
