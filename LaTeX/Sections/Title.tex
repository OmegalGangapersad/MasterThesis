\begin{titlepage}
\includegraphics{Images/UL.png}\\
\vspace{2cm}
Master's Programme in Public Adminstration, Economics and Governance
    \begin{center}
        \vspace*{1cm}
        
        {\LARGE \textbf{The long term effect of the Broad-Based Black Economic Empowerment policy on the firm performance of Johannesburg Stock Exchange-listed companies}}
        
        \vspace{0.7cm}
        by \\
        \vspace{0.7cm}
       Omegal Gangapersad
                
        \vspace{1.0cm}
    \end{center}

\noindent{
\textcolor{black}{{\bf Abstract} Black Economic Empowerment was the policy act from South Africa intervening in the South African private sector intended to incentivize firms empower Black people. The accompanied Broad-Based Black Empowerment scorecard provided firms with specific targets to comply to the empowerment of Black people. This study investigated the long term relationship between B-BBEE policy and  firm performance.
Examining the historic context and prior research it appears that the costs of B-BBEE compliance outweigh the benefits, which resulted in increasingly interventionist measures by the South African government to incentivize firms to comply. Prior research did not test the relationship between B-BBEE policy and firm performance on a long time period. This study adds to the scientific literature by investigating the long term relationship  using B-BBEE rank as a proxy for B-BBEE policy and 1, 2, 3, 4 and 5 year share price return as a proxy for firm performance over the period 2004 to 2018. 
The empirical results of this study confirms prior research. The long term relationship between B-BBEE policy and firm performance is found negative.  The results of this study could serve as an input to redesign the policy measures to create inclusive economic growth in South Africa.
}
}
\vfill
\textcolor[rgb]{0.5,0.5,0.5}{
    \begin{flushleft}
    { \small
    Master Thesis Draft\\
    14 May 2019 \\
    Supervisor: Dr. P.W. van Wijck \\
    }
    \end{flushleft}
}
      
\end{titlepage}