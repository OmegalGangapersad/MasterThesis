Under the South African Apartheid regime, the majority of the population, Black people,  had unequal to no access to resources. This oppression created enormous misallocation of resources which left the democratic post Apartheid South Africa with a momentus issue to resolve. The elected African National Congress (hereafter, ANC) of 1994 led by Nelson Mandela established a new movement of Black Economic Empowerment. Black Economic Empowerment (hereafter, BEE) and the measurement thereof, the Broad-Based Black Economic Empowerment Score (hereafter, B-BBEE score), remain relevant in the current South African political sphere. BEE was the policy act intervening in the private sector intended to incentivize firms to empower Black people by providing incentives that would increase firm performance. Thus the interests of firms would aligned with the goal to empower Black people. The problem however is that the policy’s efficacy is highly contested, the impact of BEE on firm performance is ambiguous.  The central aim of this thesis is to add clarity by providing empirical analysis on the long term effect of B-BBEE score on firm profitability.  
\section{Black Economic Empowerment and the current South African political climate}
BEE remains relevant within the social context, as the policy measure has failed to serve the common people. 2019 was another important year in South African history, as many South Africans participated in the national democratic elections on 8 May 2019 \cite[]{N8}. The ANC came under tremendous pressure amidst an avalanche of corruption cases [11]. Unlike previous national elections, voters felt less compelled to vote for the ANC. To demonstrate this, polling agency Ipsos found that as of January 2019, 39\% of respondents indicated that no political party represented their views [20].

The opposition party Economic Freedom Fighters (hereafter EFF) attested that the ANC drifted away from its original vision, cemented in the seminal Freedom Charter of 1955 [21]. The Freedom Charter called for economic freedom of the Black oppressed people. EFF leader Julius Malema accused ANC leaders, in particular South Africa’s current president Ramaphosa and former president Zuma, of betraying the goal of the Freedom Charter by accepting bribes from companies in return for political favoritism [11; 19]. In their typical provocative fashion, the EFF claimed that president Ramaphosa was a front, for the rich and powerful White people that actually run South Africa [19]. President Ramaphosa amassed great fortune, through the procurement of shares of firm previously wholly owned by White people. The Freedom Charter inspired the BEE policy which facilitated the share ownership of president Ramaphosa in Lonmin. 
\section{BEE, B-BBEE and firm performance}
BEE went through several iterations by government institutions that allowed it to evolve from a relatively loose abstract notion of BEE to specific forced guidelines of the Broad-Based Black Economic Empowerment (hereafter B-BBEE). These guidelines, called Codes of Good Practise, state specific targets that a Johannesburg Stock Exchange (JSE) listed firm should comply to, to be B-BBEE compliant. Independent auditors score firms according to their adherence to the targets set in the Codes of Good Practise. The weighted average score on each of the targets result in the aggregate B-BBEE score.

BEE, even prior to becoming government policy affected actions of South African firms. The international society banned South African companies from the international markets during the Apartheid era [24. p3]. As Apartheid was fading, and under pressure from the international ban, some South African firms started the first phase of BEE by selling shares to the Black influentials to gain goodwill by both the general public and the upcoming powerful Black elite.  Specifically, the aim of self regulations was to redistribute the capital shares of major corporations to the disadvantaged Black people [6, p13; 2, p2; 4, p17]. However, the transfer of shares only benefited a select group of well-connected politicians or business people, who were previously oppressed freedom fighters that were classified as impoverished [4 ,p5; 2, p2]. Apartheid had ended in 1991 and newly elected ANC began to correct the persisting wealth inequalities and in later years set up a BEE Commission headed by president Ramaphosa to investigate what Black Economic Empowerment policies should be instituted to reach the desired state of economic freedom for all previously disadvantaged Black people [2, p2]. The BEE Commission recommended the state creates regulations and guidelines that enforced private and state owned firms to adopt BEE policies covering a broader spectrum which would benefit all citizens [4, p19].

These regulations and guidelines were realized in the Broad-Based Black Empowerment Act in 2003 [6, p16]. The aim of this Act, as can be guessed from its name, was to provide the private sector with specific guidelines to empower Black people in a broad spectrum of initiatives, rather than share transfer to a select group. The Act targeted 7 categories of Black empowerment; ownership, management control, employment equity, skills development, preferential procurement, enterprise development and socio-economic development [7,  p545]. To offer the private sector with even further guidance, the B-BBEE Act enacted a scorecard, a weighted score of a company on each of the 7 targeted categories called the B-BBEE score [6, p16]. External companies, called B-BBEE verification agencies audited and provided a B-BBEE score of a company based on their effort on each of the 7 categories. These 7 elements were reduced to 5 elements in 2013 and effective in 2015, this change sought to be more enforcing whereby companies had to adopt the policy if they wished to provide services or goods to the government and public entities [6, p16]. Therefore, compliance of firms to B-BBEE incentivized as compliance was a prerequisite to increase firm performance by tendering for government contracts.

This study focuses on the time period between 2004 and 2018, when the B-BBEE scorecard was implemented. In 2015, Thomas Piketty had referred to B-BBEE policy and addressed South Africa’s wealth inequality and stated that 60\%-65\% of South Africa’s wealth was concentrated in the hands of the top 10\% of the population [3]. Despite the establishment of BEE and B-BBEE policy that was meant to address wealth inequalities, wealth inequality has persisted throughout these years. BEE was intended to reduce wealth inequality. The persisting wealth inequality referred to by Piketty could indicate that the incentives presented to firms were not large enough to offset the costs of the B-BBEE policy. 

\section{Drivers of the relationship between B-BBEE and firm performance}
The incentives offer to firms to comply to B-BBEE reveal the drivers behind the relationship between B-BBEE and firm performance. This study sources drivers of the relationship and categorizes these drivers in three , self-defined, non-mutually exclusive categories. Providing these categorizations, made up by the writer of this study, provides the reader with structure. The categories identified are signalling, compliance and productivity.

Although no formal definition may exist regarding signalling in the B-BBEE context, in this study signalling serves as an umbrella term for Corporate Social Responsibility (CSR) and Fronting. CSR is considered to be a positive signal whereas Fronting is associated with negative signals. Fronting was shortly referred to earlier, which entails advertizing the B-BBEE compliance to improve firm performance without actually incorporating the values and intents of B-BBEE. CSR, in contrast, does entail the culture of ‘doing good is the right thing’ and as result firm performance improves  [54, p1]. Research indicates that fronting does not improve firm performance in the long term, whereas CSR does  [34, p376; [7, p547].

Regarding compliance effects, the South African government suggests that a higher the B-BBEE level/score of improves firm performance [4, p46; 4, p10; 4, p18]. This regards access to government tenders, mentioned earlier. However, compliance also reduces firm performance as for example the transfer of shares were executed at a discount of market prices  [23, p5-p6].

The productivity effect of B-BBEE drives firm performance positively or negatively, depending on the extent to which economic renting existed in the Apartheid era and whether the extent to which market friction creates a barrier to dissolve economic renting. Put straightforward, privileged White people positioned themselves in managerial position wherein their added value to the economic process was microscopic at best. Rather, privileged White people reaped the benefits from their black subordinates hard work that did add significant value to the economic process but Black subordinates did not reap much of the benefits of their work. By implementing B-BBEE, proponents argue, this drag on firm performance would be removed.
\section{BEE and B-BBEE in academia}
Academic literature has tried to identify the success and failures of BEE and B-BBEE policy. Some researchers hypothesized that companies with higher B-BBEE scores should have higher firm performance than companies with lower B-BBEE scores [4, p19]. Herein, it is important to note that researchers have assigned different proxies of firm performance, such as profitability and share price returns of companies. According to those hypothesizing for a positive causal relationship between B-BBEE score and firm performance, the increased efficiency of the firm of empowering Black people through company policies that result in a higher B-BBEE score should enhance firm performance, as slippage caused by economic rent seekers are reduced. 

However, the body of research on B-BBEE and firm performance is not in consensus. Where Alessandri et al. [24, p20], Merwe and Ferreira [7, p545] and Mehta and Ward [27, p85] do find some positive causal effect of B-BBEE score on firm performance, Acemoglu [23, p34] and Mokgobinayane [4, p3] do not find a significant impact of B-BBEE on firm performance. Acemoglu et al. [23, p34] notes as a limitation to their investigation that the effects of B-BBEE score on firm performance should take a long time to take hold. This notion is shared by other research as well [4, p19].  Indeed, these researchers have restricted their analysis only to event studies on with narrowed focus on the share ownership transfer of BEE on share price returns with an event window of smaller than a year or ran regression analyses which analysed the impact of B-BBEE score of some measure of firm performance over the next year.
\section{Relevance of this master thesis}
This thesis aims to add to the scientific literature on the impact of B-BBEE policy on firm performance by utilizing an unique dataset which encapsulates from 2004 - 2018. As such, the objective of this thesis is to fill in the shortcoming of previous research which only measured the effect of B-BBEE on firm performance over a short time horizon. B-BBEE scores were provided by Empowerdex, the leading publisher of the top 100 B-BBEE scoring publicly traded South African firms, and Merwe. Thomson Reuters Datastream was used to access public information on company performance metrics. B-BBEE should be viewed from the larger global perspective which pushes towards Environmental Social and Governance (ESG) factors. ESG factors were first introduced in 2005, considers a corporations response to climate change, how well supply chains are managed and their responsibility for caring for their workers [14]. The corporation therefore does not merely have an obligation to its share owners but to the broader society, or stake holders as well. ESG factors encapsulate all the factors, or externalities which conventional markets fail to capture. Similar to the B-BBEE score pushed by the South African government, other governments are actively stimulating the use of ESG [28]. An analysis on the long term effect of B-BBEE score on firm performance could therefore also add to the global rising interest in stakeholder consciousness to address the negative effects caused by a myopic focus on shareholders.
\section{Research question and structure of this master thesis}
This research aims to identify the long term relationship of B-BBEE policy and firm performance. To investigate this relationship proxies are used for the two variables. B-BBEE policy, is measured as the B-BBEE rank of a firm in the Empowerdex top 100. The Empowerdex top 100 sources the top 100 from Johannesburg stock exchange listed companies. The B-BBEE rank is a measurement of compliance to the B-BBEE policy, the higher the B-BBEE rank the better a firm complies to the B-BBEE policy. According to the South African government, a firm with higher compliance to the policy should have a higher firm performance. In this study share price return is used as a proxy for firm performance. It is expected that a intrusive policy such as the B-BBEE requires a long time horizon to investigate the true effect the policy has on firm performance. Prior research used annual share price returns to investigate long term relationship. This study test the impact of B-BBEE rank on 1,2,3,4 and 5 year share price return. This thesis will aim to answer the following research question: \textbf{ "What is the long term relationship between the Broad-Based Black Economic Empowerment policy on firm performance of Johannesburg Stock Exchange-listed companies?”}  To arrive at an answer for the research question the following research sub questions were posed: “What was the long term relationship between Broad-Based Black Economic Empowerment policy on firm profitability of the Johannesburg Stock Exchange-listed companies over the period 2004 - 2018?”, “What was the long term relationship between B-BBEE policy and firm performance among the three B-BBEE policy periods?” and  “Did firms operating in a sector with a higher incentive to comply to the B-BBEE policy have higher firm performance?”

The following chapter, Contextualization, provides historical context of BEE and B-BBEE policy as well as an overview of mechanics of the policy. This explores the historical incentives of South African firms to empower Black people. The Theory chapter, will further investigate causal links between firm performance and B-BBEE by providing a framework of drivers of the relationship between B-BBEE and firm performance. These two sections conclude the inductive section of this study. The third chapter, Methodology,  covers the methodology this thesis will adopt to answer the research question from a quantitative deductive standpoint. This consists of the justification of the selection of data, hypotheses and analysis techniques. The Empirical analysis will provide the results of the systematic analysis, testing each hypothesis with robustness checks and reconciling the results with previous research. The final chapter, the conclusion, will answer the research question and share limitations of this study as well as recommendations for further research.
