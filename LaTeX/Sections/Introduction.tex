Under the South African Apartheid regime, the majority of the population, Black people,  had unequal to no access to economic resources. The Apartheid oppression created large wealth inequalities which negatively impacts the South African society today. Firms should participate in addressing this issue and participate in Black Economic Empowerment (hereafter, BEE) as part of their social responsibility, but also because it could improve their performance by for example creating a wide thriving and consuming middle class. In 2003, the South African government installed the Broad-Based Black Economic Empowerment (hereafter, B-BBEE) policy which entailed a B-BBEE scorecard by which firms could measure their compliance to Broad-Based Black Economic Empowerment. With government incentives (such as project contracts) firms would be more prone to comply to B-BBEE \cite[p546]{N7}. Additionally, institutional investors require social responsibility of the firms in which they invest
(\citeauthor{N39}, \citeyear{N39}, p58; \citeauthor{N39}, \citeyear{N39}, p65). This should indicate that firms would comply to B-BBEE to increase their performance. Where, in 2013, Chief Director in Charge of Black Economic Empowerment, Mesatywa noted that firms are “obsessed” with B-BBEE policy, and therefore the incentives of the policy was working well, however academics find a negative relationship between B-BBEE policy and firm performance (\citeauthor{N31}, \citeyear{N31}, p184; \citeauthor{N7}, \citeyear{N7}, p545;  \citeauthor{N27}, \citeyear{N27}, p85). Since 2004 to current times the B-BBEE policy become more stringent through the increasing interventions. Therefore to fairly assess the efficacy of B-BBEE policy an analysis of the long term relationship between B-BBEE policy and firm performance is required.  The central aim of this thesis is to add clarity by providing this analysis on the long term effect of B-BBEE score on firm performance. 
\section{Research background}
BEE went through several iterations by government institutions that allowed it to evolve from a relatively loose abstract notion of BEE to specific forced guidelines of the Broad-Based Black Economic Empowerment (hereafter B-BBEE). These guidelines, called Codes of Good Practise, state specific targets that Johannesburg Stock Exchange (JSE) listed firms should comply to, to be B-BBEE compliant. Independent auditors score firms according to their adherence to the targets set in the Codes of Good Practise. The weighted average score on each of the targets result in the aggregate B-BBEE score.

BEE, even prior to becoming government policy affected actions of South African firms. The international society banned South African companies from the international markets during the Apartheid era \cite[p3]{N24}. Under pressure from the international ban, some South African firms started the first phase of BEE by selling shares to the Black influentials to gain goodwill by both the general public and the upcoming powerful Black elite. However, the transfer of shares only benefited a select group of previously oppressed freedom fighters turned well-connected politicians or business people (\citeauthor{N4}, \citeyear{N4}, p5; \citeauthor{N2}, \citeyear{N2}, p2). Seeking to broaden the group of beneficiaries, government installed Broad-Based Black Empowerment Act in 2003  \cite[p16]{N6}. The aim of this Act, as can be guessed from its name, was to provide the private sector with specific target to empower Black people in a broad spectrum of initiatives, rather than share transfer to a select group. In 2013 government introduced a more stringent set of targets. Although government entities are required to comply to B-BBEE policy, private firms are not. Rather, the B-BBEE policy provides incentives for firms to comply. In 2015, Thomas Piketty referred to efficacy of the B-BBEE policy and stated that 60\%-65\% of South Africa’s wealth was concentrated in the hands of the top 10\% of the population  \cite[]{N3}. Despite the establishment of BEE and B-BBEE policy that was meant to address wealth inequalities, wealth inequality has persisted throughout these years. The persisting wealth inequality referred to by Piketty could indicate that the benefits presented to firms were not large enough to offset the costs to comply to B-BBEE policy. 
\section{B-BBEE in academia}
Academic literature has tried to identify the relationship between B-BBEE policy and firm performance. Some researchers hypothesized that companies with higher B-BBEE scores should have higher firm performance than companies with lower B-BBEE scores  \cite[p19]{N4}. Herein, it is important to note that researchers have assigned different proxies of firm performance, such as profitability and share price returns of companies. According to those hypothesizing for a positive causal relationship between B-BBEE score and firm performance, the increased efficiency of the firm empowering Black people through B-BBEE policies that result in a higher B-BBEE score should enhance firm performance, as slippage caused by economic rent seekers are reduced.

However, the body of research on B-BBEE and firm performance is not in consensus. Where Merwe and Ferreira (\citeyear{N7}, p545) and Mehta and Ward (\citeyear{N27}, p85) find a negative relationship, Acemoglu et al. (\citeyear{N23}, p32) find a non significant positive relationship and Mokgobinyane (\citeyear{N4}, p3) finds a disperse set of non significant relationships between B-BBEE and firm performance. Acemoglu et al. (\citeyear{N23}, p34) note as a limitation to their investigation that the effects of B-BBEE score on firm performance should take a long time to take hold. This notion is shared by other research as well \cite[p19]{N4}. Indeed, these studies have restricted their analysis by running regression analyses which analysed the impact of B-BBEE score of some measure of firm performance over the next year and only researched specific time periods (for example, Acemoglu et al. (\citeyear{N23}, p29) only investigate the time period  2004-2006). Finally, it is important to denote that the research indicate the cost and benefits of B-BBEE policy compliance, but are unable to distill specific cost and benefits. 
\section{Relevance of this master thesis}
This is study also explores the cost and benefits to indicate why there should be a causal relationship, but does not distill these costs and benefits to assign a specific driver of the causal relationship. Rather it analyzes the aggregate effect of B-BBEE policy compliance on firm performance. However, by uniquely obtaining B-BBEE aggregate scores from 2004 to 2018, this study adds to the existing body of research by analyzing the relationship over the entire available time period. Further, this study analyzes the effect of changing B-BBEE policy on firm performance. This provides insights not earlier captured by scientific literature. Finally, this study uniquely explores whether the incentives in B-BBEE policy cause relationship between B-BBEE policy and firm performance to be different across sectors. Therefore, by adding significant quantitative research, this study contributes clarity to academia on the relationship between B-BBEE policy and firm performance.

This information is not only valuable to the academic world, but could also find appreciation with South African policy makers. Understanding the long term relationship between B-BBEE policy and firm performance provides valuable information to strategize policy to narrow wealth inequality in South Africa.
\section{Research question and structure of this master thesis}
This research aims to identify the long term relationship of B-BBEE policy and firm performance through quantitative analysis by way of linear regression. To investigate this relationship proxies are used for the two variables. B-BBEE policy, is measured as the B-BBEE rank of a firm in the Empowerdex top 100. The Empowerdex top 100 sources the top 100 from Johannesburg stock exchange listed companies. The B-BBEE rank is a measurement of compliance to the B-BBEE policy, the higher the B-BBEE rank the better a firm complies to the B-BBEE policy. In this study share price return, obtained through Thomson Reuters Datastream, is used as a proxy for firm performance. It is expected that a intrusive policy such as the B-BBEE requires a long time horizon to investigate the true effect the policy has on firm performance. Prior research used annual share price returns to investigate long term relationship. This study test the impact of B-BBEE rank on 1,2,3,4 and 5 year share price return using linear regression on observations from 2004 to 2018. 

This thesis will aim to answer the following research question: “What is the long term relationship between the Broad-Based Black Economic Empowerment policy and firm performance of Johannesburg Stock Exchange-listed companies?”.  To arrive at an answer for the research question the following research sub questions were posed: “What was the long term relationship between Broad-Based Black Economic Empowerment policy and firm performance of the Johannesburg Stock Exchange-listed companies over the period 2004 - 2018?”, “What was the long term relationship between B-BBEE policy and firm performance among the three B-BBEE policy periods?” and  “Did firms operating in a sector with a higher incentive to comply to the B-BBEE policy have higher firm performance?”

The structure of this thesis is subservient to the goal of answering the research question. This, as pointed in the previous paragraph, demands appreciation of dynamics through time and cross-sectionally. The chapters Contextualization and Theory build the theoretical framework required to generate hypothesis and answer the research question. Specifically, the Contextualization elaborates on the dynamics through time. This explores the increasing intrusiveness of B-BBEE policy through time. The Theory chapter explores firm performance which introduce costs and benefits of B-BBEE policy. This explains why a causal relationship exists between B-BBEE policy and firm performance. The costs and benefits of B-BBEE policy also indicates that specific sectors with a higher benefit to comply to the B-BBEE policy, which relates to the cross-sectional aspect required to answer the research question. The Theory further explains how previous research measured the relationship between B-BBEE policy and firm performance, and what the findings were. Thus, the Theory and Contextualization together provide a theoretical framework that encapsulate different dimensions of the relationship and models to measure the relationship between B-BBEE policy and firm performance required to provide a comprehensive answer to the main research question. This theoretical framework also provides the expected nature of the relationship between B-BBEE policy and firm performance, in general and for the various dimensions. These expectations are formalized in hypotheses. The third chapter, Methodology, covers the methodology this thesis will adopt to answer the research question from a quantitative deductive standpoint. This consists of the justification of the research method and selection of data. The Empirical Results will provide the results of the systematic, quantitative analysis, testing each hypothesis with robustness checks and reconciling the results with previous research. The final chapter, the Conclusion, will have both a body of knowledge from the theory and empirical results on each of the dimensions of the relationship as well as the general nature of the relationship between B-BBEE policy and firm performance to conclusively answer the research question, share limitations of this study as well as provide recommendations for further research.
