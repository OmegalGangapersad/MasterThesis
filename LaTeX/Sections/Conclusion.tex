In 2003, the South African government installed the Broad-Based Black Economic Empowerment (hereafter, B-BBEE) policy which entailed a B-BBEE scorecard by which firms could measure their compliance to Broad-Based Black Economic Empowerment. The B-BBEE policy consists of costs and benefits which affect firm performance and therefore firm compliance to the B-BBEE policy. This study aimed to evaluate the long term relationship between B-BBEE policy and firm performance to provide clarity on the efficacy of the B-BBEE policy. The main research question this study posed was: “What is the long term relationship between the Broad-Based Black Economic Empowerment policy on firm performance of Johannesburg Stock Exchange-listed companies?”. 

To answer the long term relationship, this study evaluated the historical background, costs and benefits associated with compliance to B-BBEE policy and models to examine the relationship between B-BBEE policy and firm performance. This theoretical framework allowed to not only answer the general effect in the long term of B-BBEE policy compliance to firm performance, but also provided insights in the implications of the B-BBEE policy amendments as well as cross sectional implications of the government intervention through B-BBEE policy. Thus the theoretical framework provided the basis for a comprehensive evaluation to answer the research question. The historical background revealed that government intervention on firm performance through B-BBEE policy increased as targets set to comply to B-BBEE increased through time. Theory suggested that profitability was the most appropriate variable to capture firm performance. The profitability of firms was affected by compliance to the B-BBEE policy, as there were costs to be incurred to become B-BBEE compliant and benefits associated with B-BBEE compliance. The benefits of B-BBEE compliance to firms revealed that B-BBEE compliance was required to gain market access which could lead to differentiation in the long term relationship between B-BBEE policy and firm performance per sector. Due to overlapping costs and benefits the Theory concluded that the costs and benefits to B-BBEE compliance should be measured on aggregate. Previous research indicated that the aggregate costs to B-BBEE compliance, measured by regressing B-BBEE compliance with firm performance, exceeded the benefit to B-BBEE compliance. Put straightforward, the Theory found that research hinted towards a negative relationship between B-BBEE policy and firm performance.

The expectation from the Theory and Contextualization on the general long term relationship between firm policy, as well as the dynamics through time and cross-sectional were formalized into hypothesis. These hypotheses were tested in the Empirical Results using linear regressions and a bootstrap method for the time period 2004 to 2018. B-BBEE policy compliance was operationalized as treatment variable B-BBEE rank. The better the B-BBEE rank the better compliance of this firm to B-BBEE policy. Firm performance, captured through profitability was measured through share price return. To capture the long term relationship between share price returns on one, two, three, four and five years were used as the dependent variable.

The Empirical Results generally confirmed the findings from the Theory and Contextualization. Similar to what the Theory suggested, the long term relationship between B-BBEE policy and firm performance over the entire time period (2004 to 2018) was found negative. Specifically on the two, three and four years share price return time horizon, the primary regression model, Model 1, found significant negative relationships between B-BBEE rank and share price return. Therefore, in general the relationship between B-BBEE policy and firm performance was found negative, implying that the on aggregate compliance to the B-BBEE policy costs firms more than that it benefited them. Further, the Empirical Results confirmed that the negative relationship between B-BBEE policy and firm performance was most pronounced after the last amendments of the B-BBEE policy. This could indicate that increasing targets of the B-BBEE policy aggravated costs of B-BBEE policy compliance. Finally, the Empirical Results confirmed sector specific dynamics to the relationship between B-BBEE policy and firm performance. Firms operating in the Industrial sector, a sector more directly linked to government, displayed strong significant negative relationships between B-BBEE policy and firm performance. The theory indicated that B-BBEE policy demanded that government entities on dealt with firms which were B-BBEE compliant. Combining the Theory and Empirical Results, this study, speculates that the findings on the sector analysis indicates that this demand acts as a tax on firms to continue to do business with government entities.

With the Theory and Empirical Results in agreement, this thesis can confidently answer the main research question: “What is the long term relationship between the Broad-Based Black Economic Empowerment policy and firm performance of Johannesburg Stock Exchange-listed companies?”.  The long term relationship between Broad-Based Black Economic Empowerment and firm performance of Johannesburg Stock Exchange-listed companies is negative. The aggregate costs of B-BBEE compliance exceed the aggregate benefits. 
\section{Implications}
Practically speaking this study re-emphasizes that the incentives of firms to comply to B-BBEE are not sufficiently aligned with the purpose of B-BBEE. Increasing the targets to B-BBEE compliance only aggravates this misalignment. As this study is the only study that investigated the relationship on a multi year basis over the entire time period that there were B-BBEE ranking, policy makers should rest assure that these findings are fairly robust. This study could serve as an input to redesign the policy measures as the ANC, under President Cyril Ramaphosa, is dedicated to create inclusive economic growth in South Africa. 
\section{Limitations}
This study was limited on the number of firms investigated. It only captured the top 60 highest ranking B-BBEE compliant firms according to the Empowerdex top 100. Therefore, it could be that there was a sample size bias. Perhaps comparing B-BBEE compliant and non compliant firms over the entire time period would generate different results. The regression analyses per sector also contained sectors with low observations, this could affect the interpretability of the regression analyses on these sectors.
\section{Recommendations}
Further research would therefore be advised to investigate how self interest of firms could effectively be used as a tool to incentivize the empowerment of Black people and thus making the goal of the B-BBEE policy a reality. In particular, the access to government business has been stated as one of the key benefits to comply to B-BBEE policy. However, this study suggests that access to government business rather acts as a costs to Johannesburg Exchange Listed firms. Further investigation into this measure could possible provide key insights to reconfigure the B-BBEE policy to better align the policy with firm performance.
