Modern South Africa is faced with wealth inequality inherited from the Apartheid era. The ANC has been criticized for their approach to reduce this wealth inequality. This approach was the policy act of Black Economic Empowerment, which required compliance of firms to the plight of wealth inequality to succeed. This study investigated whether compliance to B-BBEE resulted in firm performance improvement in the long term. If this was the case, then firms would be incentivized to comply to B-BBEE. The long term focus of this study distinguished itself from previous work, that largely focused on short term effects of B-BBEE, despite that, this study would expect to be more appropriate in investigating the long term effects to capture the true effect of the policy on firm performance.

This study took a deductive approach that established the hypothesis that could be tested. The historical insights from the Contextualization chapter on the evolution of B-BBEE revealed that the first efforts of South African firms to empower Black people resulted from self-interest. The international society blocked firms in Apartheid South Africa from international participation, imposed sanctions and funding stops for their treatment of Black People. Faced with with a crippling economy, South African firms out of self preservation began transactions to sell their shares to Black People at discount rates to regain favor. These transactions were largely a front to portray the image of a firm committed to Black empowerment. Shares in firms were sold to influential Black people that could offer market access or political favor. These transaction were structured in a highly levered manner and protected key assets by selling ownership of non-key assets. Therefore the South African government intervened. The introduction of the B-BBEE scorecard offered companies several elements to measure their compliance to the plight of Black empowerment. However, the increasingly interventionist measures with which the South African government incentivized firms to comply to the targets of the B-BBEE scorecard suggest, according to this study, that desired commitment of firms to Black empowerment remained absent. This study suggests the cost of B-BBEE compliance outweighed the benefits for firms. This study categorized the costs and into three drivers of the relationship between B-BBEE policy and firm performance; signalling, compliance and productivity. The signalling effect of B-BBEE refers to the effect that broadcasting of B-BBEE compliance to the wider public has on firm performance. Prior research used event studies capture the short term effect of signalling and indicated mostly positive effects of signalling on share price return. Short term effects however, are difficult to use as a measure of whether a firm would be consistently incentivized to comply to B-BBEE. The compliance category dealt with the benefits and costs B-BBEE compliance offered. The advantage of B-BBEE compliance is that firms can obtain a access to market, the disadvantage are the associated costs to comply. The productivity category related to the extent to which B-BBEE compliance increased efficiency of firm. Compliance and productivity overlapped, making it difficult to assess the impact these drivers had individually. Prior studies therefore mostly focussed on investigating the relationship B-BBEE and firm performance. The majority of the studies indicate a negative relationship between B-BBEE and firm performance. These findings are difficult to assess the long term effect between B-BBEE and firm performance as time horizons of only a few years are used.

Therefore, the deductive section of this study used the time period of 2004 to 2018 and tested the relationship between B-BBEE and firm performance over several time horizons. Put straightforward, the relationship between B-BBEE rank and one, two, three, four and five year share price return was tested for the period 2004 to 2018. Several models were used to test this relationship. Two regression analyses were deployed. Model 1, the FF model, closely followed the Fama and French three factor model. The Fama and French factors were found to adequately explain share price returns, therefore these factors were used as control variables. Model 2, the MF model, followed an adaptation of the Fama and French model. This study complemented these regression models over the entire time period with a bootstrap analysis to add robustness. The results of Model 1 indicated that the relationship between B-BBEE rank and two, three and four years share price return over the time period 2004 to 2018 was significantly negative, which was confirmed by Model 2. The bootstrap model did not find any significant results over the time period 2004 to 2018. Regression tests of Model 1 over different subsections of the 2004 to 2018 time period did not materially deviate from the findings over the entire time period 2004 to 2018. Moreover, the bootstrap analysis over the Industrial sector did not find a significant results. Rather, the top ranking BBBEE compliant firms underperformed the bottom BBBEE compliant firms.

This thesis set out to answer the following research question: \textbf{“What is the long term relationship between the Broad-Based Black Economic Empowerment policy on firm performance of Johannesburg Stock Exchange-listed companies?”}.  The Contextualization and Theory indicated that the relationship between B-BBEE policy and firm performance largely depends on the incentives firms have to comply to B-BBEE. Prior research focussing on material compliance to B-BBEE suggested a negative relationship between B-BBEE policy and firm performance, indicating that the costs of B-BBEE outweigh the benefits. The empirical study of this research confirm the negative relationship. Therefore, \textbf{the answer to the research question is that the long term relationship between the Broad-Based Black Economic Empowerment policy and firm performance of Johannesburg Stock Exchange-listed companies is negative}.

Practically speaking this study reemphasizes that the incentives of firms to comply to B-BBEE are not sufficiently aligned with the purpose of B-BBEE. As this study is the only study that investigated the relationship on a multi year basis over the entire time period that there were B-BBEE ranking, policy makers should rest assure that these findings are fairly robust. This study could serve as an input to redesign the policy measures as the ANC, under President Cyril Ramaphosa, is dedicated to creating inclusive economic growth in South Africa. The historical approach of the South African government to increase intervention to incentivize firms to empower Black people does not work.

This study was limited on the number of firms investigated. It only captured the top 60 highest ranking B-BBEE compliant firms according to the Empowerdex top 100. Therefore, it could be that there was a sample size bias. Perhaps comparing B-BBEE compliant and non compliant firms over the entire time period would generate different results. Further, using the B-BBEE rank as a proxy for B-BBEE policy is appropriate to investigate the relationship with firm performance, but it does not capture the key variable of self-interest of firms narrowly. The Theory was also incapable of isolating this variable. This variable, according to this study, presents itself as key to transform B-BBEE into a successful policy. 

Further research would therefore be advised to investigate how self interest of firms could effectively be used as a tool to incentivize the empowerment of Black people and thus making the goal of the B-BBEE policy a reality. 